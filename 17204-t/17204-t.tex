%%%%%%%%%%%%%%%%%%%%%%%%%%%%%%%%%%%%%%%%%%%%%%%%%%%%%%%%%%%%%%%%%%%%%%%%%%%
%% Project Gutenberg's Researches on curves of the second order,         %%
%% by George Whitehead Hearn                                             %%
%%                                                                       %%
%% This eBook is for the use of anyone anywhere at no cost and with      %%
%% almost no restrictions whatsoever.  You may copy it, give it away or  %%
%% re-use it under the terms of the Project Gutenberg License included   %%
%% with this eBook or online at www.gutenberg.net                        %%
%%                                                                       %%
%%                                                                       %%
%% Packages and substitutions:                                           %%
%%                                                                       %%
%% book:     Basic book class. Required.                                 %%
%% amsmath:  Basic AMS math. Required.                                   %%
%% amssymb:  Basic AMS symbols. Required.                                %%
%% inputenc: Basic Accept different input encodings.                     %%
%%           Could be dispensed with by changing all                     %%
%%           ISO-8859-1-specific characters.                             %%
%% graphicx: Basic graphics for EPS images. Required.                    %%
%%                                                                       %%
%%                                                                       %%
%% Producer's Comments:                                                  %%
%%                                                                       %%
%% A very straightforward text.                                          %%
%%                                                                       %%
%%                                                                       %%
%% Things to Check:                                                      %%
%%                                                                       %%
%% Spellcheck: OK                                                        %%
%% LaCheck: OK                                                           %%
%% Lprep/gutcheck: OK                                                    %%
%% PDF pages, excl. Gutenberg boilerplate:  53                           %%
%% PDF pages, incl. Gutenberg boilerplate:  64                           %%
%% ToC page numbers: OK                                                  %%
%% Images: 6 EPS                                                         %%
%% Fonts:                                                                %%
%%                                                                       %%
%%                                                                       %%
%% Compile History:                                                      %%
%%                                                                       %%
%%                                                                       %%
%% May 05: Jim Land.                                                     %%
%%         This LaTeX document was processed with Miktex on Windows XP.  %%
%%         The Miktex compiler was run three times, until the Table of   %%
%%         Contents was complete.                                        %%
%%         DVIPS was used to create the PS file.                         %%
%%         GSView32 ver. 4.6 and Ghostscript ver. 8.50 were used to      %%
%%         produce the PDF file.                                         %%
%%         Six illustrations (Fig1 thru Fig6) are included as .eps files %%
%%         in the images directory.                                      %%
%%                                                                       %%
%% Nov 05: jt. Compiled with                                             %%
%%         e-TeX, Version 3.141592-2.2 (MiKTeX 2.4)                      %%
%%         dvipdfm, version 0.13.2c for MiKTeX 2.4                       %%
%%         latex curves.tex                                              %%
%%         latex curves.tex                                              %%
%%         latex curves.tex                                              %%
%%         dvipdfm curves.dvi                                            %%
%%                                                                       %%
%%                                                                       %%
%%                                                                       %%
%%%%%%%%%%%%%%%%%%%%%%%%%%%%%%%%%%%%%%%%%%%%%%%%%%%%%%%%%%%%%%%%%%%%%%%%%%%


\documentclass[11pt,oneside]{book}
\listfiles
\usepackage{amsmath, amssymb}
\usepackage[latin1]{inputenc}
\usepackage{graphicx}
\renewcommand{\baselinestretch}{1.1}
\renewcommand{\contentsname}{Table of Contents}
\setcounter{secnumdepth}{-1}  % no numbering of chapters or sections

\begin{document}
\thispagestyle{empty}
\small
\begin{verbatim}
Project Gutenberg's Researches on curves of the second order,
by George Whitehead Hearn

This eBook is for the use of anyone anywhere at no cost and with
almost no restrictions whatsoever.  You may copy it, give it away or
re-use it under the terms of the Project Gutenberg License included
with this eBook or online at www.gutenberg.net


Title: Researches on curves of the second order

Author: George Whitehead Hearn

Release Date: December 1, 2005 [EBook #17204]

Language: English

Character set encoding: TeX

*** START OF THIS PROJECT GUTENBERG EBOOK RESEARCHES ON CURVES ***




Produced by Joshua Hutchinson, Jim Land and the Online
Distributed Proofreading Team at http://www.pgdp.net.
This file was produced from images from the Cornell
University Library: Historical Mathematics Monographs
collection.



\end{verbatim}
\normalsize
\newpage


\frontmatter

\begin{titlepage}
\begin{center}
\begin{Large}
RESEARCHES ON CURVES\\
\end{Large}
\medskip
\textsc{of the}\\
\medskip
\begin{large}
SECOND ORDER,\\
\end{large}
\medskip
\begin{small}
\textsc{also on}\\
\end{small}
\medskip
\begin{large}
Cones and Spherical Conics treated Analytically,\\
\end{large}
\medskip
\begin{small}
\textsc{in which}\\
\medskip
\textsc{THE TANGENCIES OF APOLLONIUS ARE INVESTIGATED,\\
AND GENERAL GEOMETRICAL CONSTRUCTIONS\\
DEDUCED FROM ANALYSIS;}\\
\medskip
\textsc{also several of}\\
\medskip
\textsc{THE GEOMETRICAL CONCLUSIONS OF M.\ CHASLES\\
ARE ANALYTICALLY RESOLVED,}\\
\medskip
\textsc{together with}\\
\medskip
\textsc{MANY PROPERTIES ENTIRELY ORIGINAL.}
\end{small}

\bigskip
\bigskip
\textsc{by}\\
GEORGE WHITEHEAD HEARN,\\
\textsc{a graduate of cambridge, and a professor of mathematics\\
in the royal military college, sandhurst.}

\bigskip
\rule{100pt}{0.25mm}
\bigskip

\textsc{london:\\
george bell, 186, fleet street.\\
mdcccxlvi.}


\end{center}
\end{titlepage}
\pagestyle{empty}

\tableofcontents


\newpage
\mainmatter
\pagestyle{plain}

\begin{center}
\section{PREFACE.}
\end{center}


In this small volume the reader will find no
fantastical modes of applying Algebra to
Geometry. The old Cartesian or co-ordinate
system is the basis of the whole method---and
notwithstanding this, the author is satisfied
that the reader will find much originality
in his performance, and flatters himself that
he has done something to amuse, if not to
instruct, Mathematicians.

Though the work is not intended as an
elementary one, but rather as supplementary
to existing treatises on conic sections, any
intelligent student who has digested Euclid,
and the usual mode of applying Algebra to
Geometry, will meet but little difficulty in the
following pages.\\


\noindent\textit{Sandhurst},\\
30\textit{th June}, 1846.

\newpage
\addcontentsline{toc}{subsubsection}{Introductory Discourse Concerning Geometry.}

\begin{center}
\section{INTRODUCTORY DISCOURSE CONCERNING GEOMETRY.}
\end{center}

The ancient Geometry of which the Elements of
Euclid may be considered the basis, is undoubtedly
a splendid model of severe and accurate reasoning.
As a logical system of Geometry, it is perfectly
faultless, and has accordingly, since the restoration
of letters, been pursued with much avidity by many
distinguished mathematicians. Le P�re Grandi,
Huyghens, the unfortunate Lorenzini, and many
Italian authors, were almost exclusively attached
to it,---and amongst our English authors we may
particularly instance Newton and Halley. Contemporary
with these last was the immortal Des
Cartes, to whom the analytical or modern system
is mainly attributable. That the complete change
of system caused by this innovation was strongly
resisted by minds of the highest order is not at all
to be wondered at. When men have fully recognized
a system to be built upon irrefragable truth,
they are extremely slow to admit the claims of any
different system proposed for the accomplishment of
the same ends; and unless undeniable advantages
can be shown to be possessed by the new system,
they will for ever adhere to the old.

But the Geometry of Des Cartes has had even
more to contend against. Being an instrument of
calculation of the most refined description, it requires
very considerable skill and long study before
the student can become sensible of its immense
advantages. Many problems may be solved in
admirably concise, clear, and intelligible terms by
the ancient geometry, to which, if the algebraic
analysis be applied as an instrument of investigation,
long and troublesome eliminations are met
with,\footnote{This however is usually the fault of the \textit{analyst} and not of
the \textit{analysis}.} and the whole solution presents such a contrast
to the simplicity of the former method, that a
mind accustomed to the ancient system would be
very liable at once to repudiate that of Des Cartes.
On the other hand, it cannot be denied that the
Cartesian system always presents its results as at
once derived from the most elementary principles,
and often furnishes short and elegant demonstrations
which, according to the ancient method,
require long and laborious reasoning and frequent
reference to propositions previously established.

It is well known that Newton extensively used
algebraical analysis in his geometry, but that, perhaps
partly from inclination, and partly from compliance
with the prejudice of the times, he translated
his work into the language of the ancient
geometry.

It has been said, indeed (vide Montucla, part~V.\ 
liv.~I.), that Newton regretted having passed too
soon from the elements of Euclid to the analysis of
Des Cartes, a circumstance which prevented him
from rendering himself sufficiently familiar with
the ancient analysis, and thereby introducing into
his own writings that form and taste of demonstration
which he so much admired in Huyghens and the
ancients. Now, much as we may admire the logic
and simplicity of Euclidian demonstration, such
has been the progress and so great the achievements
of the modern system since the time of
Newton, that there seems to be but one reason why
we may consider it fortunate that the great ``Principia''
had previously to seeing the light been translated
into the style of the ancients, and that is, that
such a style of geometry was the only one then
well known. The Cartesian system had at that
time to undergo its ordeal, and had the sublime
truths taught in the ``Principia'' been propounded
and demonstrated in an almost unknown and certainly
unrecognised language, they might have lain
dormant for another half century. Newton certainly
was attached to the ancient geometry (as who
that admires syllogistic reasoning is not?) but he
was much too sagacious not to perceive what an
instrument of almost unlimited power is to be found
in the Cartesian analysis if in the hands of a
skilful operator.

The ancient system continued to be cultivated in
this country until within very recent years, when
the Continental works were introduced by Woodhouse
into Cambridge, and it was then soon seen
that in order to keep pace with the age it was absolutely
necessary to adopt analysis, without, however,
totally discarding Euclid and Newton.

We will now advert to an idea prevalent even
amongst analysts, that analytical reasoning applied
to geometry is less rigorous or less instructive than
geometrical reasoning. Thus, we read in Montucla:
``La g�om�trie ancienne a des avantages qui
feroient desirer qu'on ne l'eut pas autant abandonn�e.
Le passage d'une v�rit� � l'autre y est
toujours clair, et quoique souvent long et laborieux,
il laisse dans l'esprit une satisfaction que ne donne
point le calcul alg�brique qui convainct sans
�clairer.''

This appears to us to be a great error. That a
young student can be sooner taught to comprehend
geometrical reasoning than analytical seems natural
enough. The former is less abstract, and deals
with tangible quantities, presented not merely to
the mind, but also to the eye of the student. Every
step concerns some line, angle, or circle, visibly
exhibited, and the proposition is made to depend on
some one or more propositions previously established,
and these again on the axioms, postulates,
and definitions; the first being self-evident truths,
which cannot be called in question; the second
simple mechanical operations, the possibility of
which must be taken for granted; and the third
concise and accurate descriptions, which no one can
misunderstand. All this is very well so far as it
goes, and is unquestionably a wholesome and excellent
exercise for the mind, more especially that
of a beginner. But when we ascend into the
higher geometry, or even extend our researches in
the lower, it is soon found that the \textit{number of propositions}
previously demonstrated, and on which
any proposed problem or theorem can be made to
depend, becomes extremely great, and that demonstration
of the proposed is always the best which
combining the requisites of conciseness and elegance,
is at the same time the most elementary, or
refers to the fewest previously demonstrated or
known propositions, and those of the simplest kind.
It does not require any very great effort of the mind
to remember all the propositions of Euclid, and how
each depends on all or many preceding it; but
when we come to add the works of Apollonius, Pappus,
Archimedes, Huyghens, Halley, Newton, \&c.,
that mind which can store away all this knowledge
and render it available on the spur of the
moment is surely of no common order. Again, the
moderns, Euler, Lagrange, D'Alembert, Laplace,
Poisson, \&c., have so far, by means of analysis,
transcended all that the ancients ever did or thought
about, that with one who wishes to make himself
acquainted with their marvellous achievements it is
a matter of imperative necessity that he should
abandon the ancient for the modern geometry, or at
least consider the former subordinate to the latter.
And that at this stage of his proceeding he should
by no means form the very false idea that the modern
analysis is less rigorous, or less convincing, or less
instructive than the ancient syllogistic process. In
fact, ``more'' or ``less rigorous'' are modes of expression
inadmissible in Geometry. If anything is
``less rigorous'' than ``absolutely rigorous'' it is
no demonstration at all. We will not disguise the
fact that it requires considerable patience, zeal, and
energy to acquire, \textit{thoroughly understand}, and retain
a system of analytical geometry, and very frequently
persons deceive themselves by thinking that
they fully comprehend an analytical demonstration
when in fact they know very little about it. Nay
it is not unfrequent that people write upon the subject
who are far from understanding it. The cause
of this seems to be, that such persons, when once
they have got their proposition translated into
equations, think that all they have then to do is to
go to work \textit{eliminating} as fast as possible, without
ever attempting any \textit{geometrical} interpretation of any
of the steps until they arrive at the final result.
Far different is the proceeding of those who fully
comprehend the matter. To them every step has a
geometrical interpretation, the reasoning is complete
in all its parts, and it is not the least recommendation
of the admirable structure, that it is composed of
only a few elementary truths easily remembered, or
rather impossible to be forgotten.




\begin{center}
\chapter{CHAPTER I.}
\end{center}

\addcontentsline{toc}{section}{Problem proposed by Cramer to Castillon}
It is intended in this chapter to apply analysis to
some problems, which at first view do not seem to
be susceptible of concise analytical solutions, and
which possess considerable historical interest. The
first of these is one proposed by M. Cramer to M.
de Castillon, and which may be enunciated thus:
``Given three points and a circle, to inscribe in the
circle a triangle whose sides shall respectively pass
through the given points.''

Concerning this curious problem Montucla remarks
that M. de Castillon having mentioned it to
Lagrange, then resident at Berlin, this geometer
gave him a purely analytical solution of it, and
that it is to be found in the Memoirs of the Academy
of Berlin (1776), and Montucla then adds, ``Elle
prouve � la fois la sagacit� de son auteur et les ressources
de notre analyse, mani�e par d'aussi habiles
mains.'' Not having the means of consulting the
Memoir referred to, I have not seen Lagrange's solution,
nor indeed any other, and as it has been considered
a difficult problem I have considered it a fit
subject to introduce into this work as an illustration
of the justness of the remarks made in the introductory
discourse.

The plan I have adopted is the following:---

%[Illustration]
\begin{figure}[htb]
\begin{center}
\includegraphics*[width=12cm]{images/fig1}
\end{center}
\end{figure}

Let A,~B,~C be the given points. Draw a pair of
tangents from A, and let PQH be the line of contact.
Similarly pairs of tangents from B and C,
SRK, VTL being lines, of contact. Then if a
triangle KLH can be described about the circle,
and such that its angular points may be in the given
lines PQH, SRK, VTL respectively, then the
points of contact, X,~Y,~Z being joined will pass respectively
through A,~B,~C. For H being the pole
of ZX, tangents drawn where any line HQP intersects
the circle will intersect in ZX produced,
but those tangents intersect in A, and therefore
ZX passes through A. Similarly of the rest.

When any of the points A,~B,~C falls within the
circle as at $a$, join $oa$. Make $ap + oam$, and
draw tangent $pm$, then $\mathrm{A}m\mathrm{H} \perp om$ will hold the
place of PQH in the above.

We have therefore reduced the problem to the
following.

Let there be three given straight lines and a given
circle, it is required to find a triangle circumscribed
about the circle, which shall have its angular points
each in one of the three lines.

Let $a$ be the radius of the circle, and let the
equations to the required tangents be

\[
\left.
  \begin{array}{l}
    \l_1 x + m_1 y = a \\
    \l_2 x + m_2 y = a \\
    \l_3 x + m_3 y = a
  \end{array}
\right\} \tag{1}
\]
Also the equations to the three given lines
\[
\left.
\begin{aligned}
  \mathrm{A}_1 x + \mathrm{B}_1 y &= p_1  \\
  \mathrm{A}_2 x + \mathrm{B}_2 y &= p_2  \\
  \mathrm{A}_3 x + \mathrm{B}_3 y &= p_3
\end{aligned}
\right\}
\tag{2}
\]
$p_1$,~$p_2$,~$p_3$ being perpendiculars upon them from the
centre of the circle, and $l_1$,~$m_1$,~$\mathrm{A}_1$,~$\mathrm{B}_1$ \&c., direction
cosines.

Suppose the intersection of the two first lines of
(1) to be in the third line of (2), we have by eliminating
$x$ and $y$ between
\begin{align*}
  l_1 x + m_1 y &= a  \\
  l_2 x + m_2 y &= a  \\
\text{and } \mathrm{A}_3 x + \mathrm{B}_3 y &= p_3
\end{align*}
the condition
\begin{align*}
  \mathrm{A}_3 a(m_2 - m_1) + \mathrm{B}_3 a(l_1 - l_2) &= p_3(l_1 m_2 - l_2 m_1)  
\intertext{and similarly }
  \mathrm{A}_2 a(m_1 - m_3) + \mathrm{B}_2 a(l_3 - l_1) &= p_2(l_3 m_1 - l_1 m_3)  \\
  \mathrm{A}_1 a(m_3 - m_2) + \mathrm{B}_1 a(l_2 - l_3) &= p_1(l_2 m_3 - l_3 m_2)
\end{align*}  
\begin{flalign*}
&\text{Now let }
&l_1 &= \cos \theta_1,\quad \therefore m_1 = \sin \theta_1 \text{ \&c.}  &&\\
&\text{Also }
&\mathrm{A}_3 &= \cos \alpha_3,\quad \therefore \mathrm{B}_3 = \sin \alpha_3 \text{ \&c.}  &&
\end{flalign*}
Then the first of the above conditions is
\[
  a\{ \cos \alpha_3 (\sin \theta_2 - \sin \theta_1)
    + \sin \alpha_3 (\cos \theta_1 - \cos \theta_2) \}
  = p_3 \sin (\theta_2 - \theta_1)
\]
This equation is easily reducible by ordinary trigonometry to
\[
\tan\frac{\alpha_3-\theta_1}{2} \tan\frac{\alpha_3-\theta_2}{2} +
\frac{p_3-a}{p_3+a} = 0
\]
Similarly
\begin{align*}
\tan \frac{\alpha_2-\theta_3}{2} \tan\frac{\alpha_2-\theta_1}{2} +
\frac{p_2-a}{p_2+a} &= 0
\\
\tan \frac{\alpha_1-\theta_2}{2} \tan\frac{\alpha_1-\theta_3}{2} +
\frac{p_1-a}{p_1+a} &= 0
\end{align*}

If now for brevity we put
$x = \tan\dfrac{\theta_1}{2}$,
 $y = \tan\dfrac{\theta_2}{2}$,
$z=\tan\dfrac{\theta_3}{2}$, also\footnote{PG proofer's note:
In the original, the numerator and denominator of $k_3$ are identical.} 
$k_3 =\dfrac{p_3+a\cos \alpha_3}{p_3+a\cos \alpha_3}$,
$h_3 = \dfrac{a\sin \alpha_3}{p_3-a \cos \alpha_3}$
\&c. the above equations become
\begin{align*}
k_3xy -h_3(x+y)+1 & = 0 \\
k_2zx -h_2(z+x)+1 & = 0 \\
k_1yz -h_1(y+z)+1 & = 0
\end{align*}
from which we can immediately deduce a quadratic
for $x$.

On eliminating $z$ between the second and third
equations, we shall have another equation in $x$ and $y$
similar in form to the first.

We may, moreover, so assume the axis from which
$\alpha_1$, $\alpha_2$, $\alpha_3$ are measured, so that $h_3 = 0$ and the
equations are then,
\[
k_3 x y + 1 = 0
\]
\[
\text{and } \qquad (h_2 k_1 - h_1 k_2) x y + (h_1 h_2 - k_1) y -
(h_1 h_2 - k_2) x + h_2 - h_1 = 0.
\]

These are, considering $x$ and $y$ as co-ordinates,
the equations to two hyperbolas having parallel
asymptotes, and which we may assume to be rectangular.
To show that their intersections may be
easily determined geometrically, assume the equations
under the form
\begin{align*}
&x y = \mathrm{C}^2  \\
x y - \mathrm{C}^2 + &\mu \left( \frac{x}{\mathrm{A}} + \frac{y}{\mathrm{B}} - 1 \right) = 0
\end{align*}

Then by subtraction,
\[
\frac{x}{\mathrm{A}} + \frac{y}{\mathrm{B}} - 1  = 0
\]
is the common secant.

\begin{flalign*}
&\text{\indent Let } &\frac{x}{\mathrm{B}} + \frac{y}{\mathrm{A}} - 1  &= 0  &&
\end{flalign*}
be another secant.

Multiply these together and we have
\[
\frac{x^2 + y^2}{\mathrm{AB}} + \frac{\mathrm{A}^2 + \mathrm{B}^2}{\mathrm{A}^2\mathrm{B}^2} x y
 + \frac{\mathrm{A}+\mathrm{B}}{\mathrm{AB}} (x+y) + 1 = 0
\]

This equation represents the \textit{two} secants. But at
the points of their intersection with the hyperbola
$xy= \mathrm{C}^2$, this last equation reduces to
\begin{align*}
  x^2+y^2-&(\mathrm{A}+\mathrm{B})(x+y)+\mathrm{A}\mathrm{B}
+\mathrm{C}^2\frac{\mathrm{A}^2+\mathrm{B}^2}{\mathrm{A}\mathrm{B}}=0\\
\text{or}\quad &\left(x-\frac{\mathrm{A}+\mathrm{B}}{2}\right)^2
+\left(y-\frac{\mathrm{A}+\mathrm{B}}{2}\right)^2\\
&=(\mathrm{A}^2+\mathrm{B}^2)\left\{\frac{\mathrm{A}\mathrm{B}-
\mathrm{C}^2}{2\mathrm{A}\mathrm{B}}\right\}
\end{align*}
which represents a circle, co-ordinates of the centre
\begin{align*}
  &x = y = \frac{\mathrm{A} + \mathrm{B}}{2} \\
  \text{and radius } \quad
&(\mathrm{A}^2 + \mathrm{B}^2)^\frac{1}{2}
  \left\{ \frac{\mathrm{AB}-\mathrm{C}^2}{2\mathrm{AB}} \right\}^\frac{1}{2}
\end{align*}

Hence it is evident that A and B, being once
geometrically assigned, the rest of the construction
is merely to draw this circle, which will intersect
$\dfrac{x}{\mathrm{A}} + \dfrac{y}{\mathrm{B}} - 1 = 0$ in the required points.

The analytical values of A, B, and C$^2$ are
\[
\begin{aligned}
\mathrm{A} &= - \frac{(h_1-h_2)k_3 + h_2 k_1 - h_1 k_2}{(h_1 h_2 - k_2 )k_3}  \\
\mathrm{B} &=   \frac{(h_1-h_2)k_3 + h_2 k_1 - h_1 k_2}{(h_1 h_2 - k_1 )k_3}\\
\mathrm{C}^2 &= -\frac{1}{k_3}
\end{aligned}\;
\]
These being rational functions of known geometrical
magnitudes, are of course assignable
\textit{geometrically}, so that every difficulty is removed,
and the mere labour of the work remains.

\addcontentsline{toc}{section}{Tangencies of Apollonius}
In the next place, I propose to derive a general
mode of construction for the various cases of the
``tangencies'' of Apollonius from analysis. The
general problem may be stated thus: of three
points, three lines and three circles, any three
whatever being given, to describe another circle
touching the given lines and circles and passing
through the given points.

It is very evident that all the particular cases
are included in this, ``to describe a circle touching
three given circles,'' because when the centre of a
circle is removed to an infinite distance, and its
radius is also infinite, that circle becomes at all
finite distances from the origin a straight line.
Also, when the radius of a circle is zero it is reduced
to a point.

We will therefore proceed at once to the consideration
of this problem, and it is hoped that the
construction here given will be found more simple
than any hitherto devised.

The method consists in the application of the
two following propositions.

If two conic sections have the same focus, lines
may be drawn through the point of intersection of
their \textit{citerior} directrices,\footnote{By
the term ``citerior'' I mean those directrices nearest to
the common focus.}
and through two of the
points of intersection of the curves.

Let $u$ and $v$ be linear functions of $x$ and $y$, so
that the equations $u = 0$, $v = 0$ may represent the
citerior directrices, then if $r = \sqrt{x^2 + y^2}$, and $m$
and $n$ be constants, we have for the equations of
the two curves
\begin{align*}
  r &= m u \\
  r &= n v
\end{align*}
and by eliminating $r$, $mu - nv = 0$; but this is
the equation to a straight line through the intersection
of $u = 0$, $v = 0$, since it is satisfied by these
simultaneous equations.

When the curves are both ellipses they can intersect
only in two points, and the above investigation
is fully sufficient. But when one or both the curves
are hyperbolic, we must recollect that only one
branch of each curve is represented by each of the
above equations. The other branches are,
\begin{align*}
  r &= -mu \\
  r &= -nv
\end{align*}

We have therefore, in this instance $mu + nv = 0$
as well as $mu - nv = 0$, for a line of intersection.

The second proposition is, having given the
focus, citerior directrix, and eccentricity of a conic
section, to find by geometrical construction the two
points in which the conic section intersects a given
straight line.

In either of the diagrams, the first of which is for
an ellipse, the second for a hyperbola, let MX be
the given straight line, F the focus, A
the vertex,
and DR the citerior directrix. Let
\[
  \mathrm{FM + MX} = p, \quad \mathrm{MFD} = \alpha,
\]
$r$ the distance of any point in MX from F,
$\theta$ the
angle it makes with FD, and $\mathrm{FD} = a$.
Also let $n = \dfrac{\mathrm{FA}}{\mathrm{AD}}$

Then \quad $\dfrac{r}{a-r \cos \theta}= n, \quad r \cos (\theta - \alpha) = p$\\
\\
Eliminate $r$
\[
  \frac{a}{p} \cos (\theta - \alpha) - \cos \theta = \frac{1}{n}
\]
or
\[
  (a \cos \alpha - p) \cos \theta + a \sin \alpha \sin \theta = \frac{p}{n}
\]
Let
\[
  \frac{a \cos \alpha - p}{a \sin \alpha} = \cot \varepsilon
\tag{1}
\]
Then
\[
  \frac{a \sin \alpha}{\sin \varepsilon}
  \cos (\theta \sim  \varepsilon) = \frac{p}{n}
\]
or
\[
  \cos (\theta \sim \varepsilon)
= \frac{p \sin \varepsilon}{n a \sin \alpha} = \frac{p}{nd}
\tag{2}
\]
where $d = \dfrac{a \sin \alpha}{\sin \varepsilon}$

\medskip
From the formul� (1) and (2) we derive the following
construction. Join DM, then DMH is
the angle $\varepsilon$, because DM projected on FH is \smallskip
$a \cos \alpha - p$. Also a perpendicular from D on FH
is $a \sin \alpha$,
$\therefore \dfrac{a \cos \alpha - p}{a \sin \alpha} = \cot \mathrm{DMH}$,
$\therefore \cot \mathrm{DMH} = \cot \varepsilon$,
$\therefore \mathrm{DMH} = \varepsilon$ \medskip

%[Illustration]
\begin{figure}[htb]
\begin{center}
\includegraphics*[width=10cm]{images/fig2}
\end{center}
\end{figure}

Again, $ \dfrac{\sin \mathrm{DMF}}{\sin \mathrm{DFM}} = \dfrac{\mathrm{DF}}{\mathrm{DM}} $, or \smallskip
$\dfrac{\sin \varepsilon}{\sin \alpha} = \dfrac{a}{\mathrm{DM}}$, whence
$\mathrm{DM} = d$. Find ML a third proportional to
AD, FA and DM, so that $\mathrm{ML} = nd$. With
centre M and radius ML describe a circle. Make
MH equal to FM, and draw KHL at right angles
to FH, and join MK, ML\@. Then by (2) LMH
or $\mathrm{KMH} = \theta \sim \varepsilon$.

%[Illustration]
\begin{figure}[htb]
\begin{center}
\includegraphics*[width=9cm]{images/fig3}
\end{center}
\end{figure}

Taking the value $\varepsilon - \theta$, we have therefore 
\[
\mathrm{LMD}= \mathrm{DMH} - \mathrm{LMH} = \varepsilon -(\varepsilon - \theta) = \theta.
\] 
And taking $\theta - \varepsilon$, we have 
\[
\mathrm{KMD} = \mathrm{KMH} + \mathrm{HMD} = \varepsilon + \theta - \varepsilon = \theta.
\]

Hence, make $\mathrm{QFX} = \mathrm{LMD}$, $\mathrm{PFD} = \mathrm{KMD}$,
and P and Q are the two points required.

We now proceed to show how, by combining
these two propositions, the circles capable of simultaneously
touching three given circles may be found.

Let A, B, C, be the centres of the three circles,
and let the sides of the triangle ABC be as usual
denoted by $a$,~$b$,~$c$; the radii of the circles being
$\alpha$,~$\beta$,~$\gamma$.

We will suppose that the circle required \textit{envelopes}
A and touches B and C externally, and the same
process, \textit{mutatis mutandis}, will give the other
circles.

Taking AB for axis of $x$, and A for origin, we
easily find in the usual way the equation to the
hyperbola, which is the locus of the centres of the
circles touching A and B.
\[
  r = \frac{c^2-(\alpha + \beta)^2 - 2cx}{2(\alpha + \beta)}
\]

%[Illustration]
\begin{figure}[htb]
\begin{center}
\includegraphics*[width=8cm]{images/fig4}
\end{center}
\end{figure}


From which, DF being the citerior directrix, we
have
\[
\mathrm{AD} = \dfrac{c^2-(\alpha + \beta)^2}{2c}
\]
Hence, with radius
$\mathrm{BK} = \alpha + \beta$ describe an arc. Bisect AB, and
from its middle point as centre and rad.\ $\tfrac{1}{2} \mathrm{AB}$
describe an arc, intersecting the former in K. Draw
$\mathrm{KN} \perp \mathrm{AB}$, and bisect AN in D, then $\mathrm{DF}
\perp \mathrm{AB}$ is the citerior directrix. Again, make AV
to AD as $c$ to $\alpha + \beta + c$, \textit{i.e.} as AB to rad.~A + rad.~B + AB, and V will be the citerior
vertex.

Assign the citerior directrix EF of the hyperbola,
which is the locus of the circles touching
A and C. Make DG to EH in the ratio
compounded of the ratios of $b$ to $c$, and $\alpha + \beta$
to $\alpha + \gamma$. Draw GS and HS $\perp$ to BA and
CA, and through S and F draw SPFQ; this
will be the line of centres, and by applying the
second proposition, two points, P and Q, will be
found. Join PA, and produce it to meet the
circle A in L, and with radius PL describe a circle,
and this will envelope A and touch B and C externally.
Also, if QA be joined, cutting circle A in
$\mathrm{L}'$, and a circle radius $\mathrm{QL}'$ be described, it will
envelope B and C, and touch A externally.

Similarly the three other pairs of circles may be
found.

As it would too much increase the extent of this
work to go \textit{seriatim} through the several cases of the
tangencies---that is, to apply the foregoing propositions
to each case, the reader is supposed to apply
them himself.

I have in the ``Mathematician,'' vol.~I, p.~228,
proposed and proved a curious relation amongst
the radii of the eight tangent circles. The following
is another curious property.

\addcontentsline{toc}{section}{Curious property respecting the directions of hyperbol�; which are
the loci of centres of circles touching each pair of three circles.}
With reference to the last figure, suppose we
denote the hyperbolic branch of the locus of the
centres of circles enveloping A and touching B
externally by $\mathrm{A}_c\;\mathrm{B}_u$, $\mathrm{A}_u\;\mathrm{B}_c$, the former meaning
``branch citerior to A and ulterior to B,'' the latter
``citerior to B and ulterior to A.'' The six hyperbolic
branches will then be thus denoted:
\[
  \mathrm{A}_c\;\; \mathrm{B}_u ,\;\; \mathrm{A}_u\;\; \mathrm{B}_c ;\;\;
  \mathrm{B}_c\;\; \mathrm{C}_u ,\;\; \mathrm{B}_u\;\; \mathrm{C}_c ;\;\;
  \mathrm{C}_c\;\; \mathrm{A}_u ,\;\; \mathrm{C}_u\;\; \mathrm{A}_c
\]
and suppose the corresponding directrices denoted
thus:
\[
  \overline{ \mathrm{A}_c\;\; \mathrm{B}_u } ,\;\;
  \overline{ \mathrm{A}_u\;\; \mathrm{B}_c } ;\;\;
  \overline{ \mathrm{B}_c\;\; \mathrm{C}_u } ,\;\;
  \overline{ \mathrm{B}_u\;\; \mathrm{C}_c } ;\;\;
  \overline{ \mathrm{C}_c\;\; \mathrm{A}_u } ,\;\;
  \overline{ \mathrm{C}_u\;\; \mathrm{A}_c }
\]
Then the point P is the mutual intersection of
\[
  \mathrm{A}_c \; \mathrm{C}_u ,\;
  \mathrm{A}_c \; \mathrm{B}_u ,\;
  \mathrm{B}_c \; \mathrm{C}_u
\]
and Q is the mutual intersection of
\[
  \mathrm{A}_u \; \mathrm{C}_c ,\;
  \mathrm{A}_u \; \mathrm{B}_c ,\;
  \mathrm{B}_u \; \mathrm{C}_c
\]

PQ passes through intersection of $\overline{\mathrm{A}_c \mathrm{B}_u}$,
$\overline{\mathrm{A}_c \mathrm{C}_u}$
because it passes through intersections of $\mathrm{A}_c \mathrm{B}_u$,
$\mathrm{A}_c \mathrm{C}_u$, and of $\mathrm{A}_u \mathrm{C}_c$, $\mathrm{A}_u \mathrm{B}_c$.

Also PQ through $\overline{\mathrm{B}_c \mathrm{A}_u}$,
 $\overline{\mathrm{B}_c \mathrm{C}_u}$, because through
$\mathrm{B}_c \mathrm{A}_u$, $\mathrm{B}_u \mathrm{C}_c$ and
 $\mathrm{B}_u \mathrm{A}_c$, $\mathrm{B}_c \mathrm{C}_u$.

Also PQ through $\overline{\mathrm{C}_c \mathrm{A}_u}$,
$\overline{\mathrm{C}_c \mathrm{B}_u}$, because through
$\mathrm{C}_c \mathrm{A}_u$, $\mathrm{C}_c \mathrm{B}_u$ and
$\mathrm{C}_u \mathrm{A}_c$, $\mathrm{C}_u \mathrm{B}_c$, and hence the intersections
$\overline{\mathrm{A}_c \mathrm{B}_u}$, $\overline{\mathrm{A}_c \mathrm{C}_u}$;
$\overline{\mathrm{B}_c \mathrm{A}_u}$, $\overline{\mathrm{B}_c \mathrm{C}_u}$;
$\overline{\mathrm{C}_c \mathrm{A}_u}$, $\overline{\mathrm{C}_c \mathrm{B}_u}$ are all in
the same straight line PQ.

That is, the intersections of pairs of directrices
citerior respectively to A,~B,~C are in the same
straight line, namely, the line of centres of the pair
of tangent circles to which they belong.




\begin{center}
\chapter{CHAPTER II.}
\end{center}


On curves of the second order passing through
given points and touching given straight lines. \medskip

Let $u=0$, $v=0$, $w=0$, be the equations to three
given straight lines.

The equation
\[
\lambda vw+\mu uw+\nu uv=0 \tag{1}
\]
being of the second order represents a conic section,
and since this equation is satisfied by any two of
the three equations $u=0$, $v=0$, $w=0$, (1) will pass
through the three points formed by the mutual
intersections of those lines.

To assign values of $\lambda$,~$\mu$,~$\nu$, in terms of the co-ordinates
of the centre of~(1),

We have
\begin{align*}
u &= a_2x+b_2y+1 \\
v &= a_3x+b_3y+1 \\
w &= a_4x+b_4y+1
\end{align*}
Hence (1) differentiated relatively to $x$ and $y$ will
give
\[
\begin{aligned}
\lambda\{a_4v+a_3w\}+\mu\{a_2w+a_4u\}+\nu\{a_3u+a_2v\}&=0\\
\lambda\{b_4v+b_3w\}+\mu\{b_2w+b_4u\}+\nu\{b_3u+b_2v\}&=0
\end{aligned}
\tag{a}
\]
and these are the equations for finding the co-ordinates
of the centre.

Let now L,~M,~and~N be three such quantities that
\[
\mathrm{L}u+\mathrm{M}v+\mathrm{N}w
\]
may be \textit{identically} equal to 2K, then by finding
the ratios $\dfrac{\lambda}{\mu}$,~$\dfrac{\lambda}{\nu}$ \smallskip from ($a$) it will be found that the
following values may be assigned to $\lambda$,~$\mu$,~$\nu$,
\[
\lambda = u (Lu - K), \quad \mu = v (Mv - K), \quad \nu = w (Nw - K)
\]
Hence when any relation exists amongst $\lambda$,~$\mu$,~$\nu$, we
can, by the substitution of these values, immediately
determine the locus of the centres of (1).
\addcontentsline{toc}{section}{Locus of centres of all conic sections through same four points}

\bigskip
$1^\textrm{o}$ Let (1) pass through a fourth point, then $\lambda$,~$\mu$,~$\nu$,
are connected by the relation
\[
\mathrm{A} \lambda + \mathrm{B} \mu + \mathrm{C} \nu = 0 \tag{$\alpha$}
\]
where A,~B,~C are the values of $vw$,~$uw$,~$uv$, for
the fourth point.

Hence the locus of the centres of all conic sections
drawn through the four points will be
\[
\mathrm{A} u (\mathrm{L}u - \mathrm{K}) + \mathrm{B} v (\mathrm{M}v -
\mathrm{K}) + \mathrm{C} w (\mathrm{N}w - \mathrm{K}) = 0  \tag{\textit{b}}
\]
which is itself a curve of the second order.
\addcontentsline{toc}{section}{Locus of centres of all conic sections through
 two given points, and touching a given line in a given point}

\bigskip
$2^\textrm{o}$ When the fourth point coincides with one of
the other points, the values of A,~B,~C vanish. But
suppose the fourth point infinitely near to the intersection
of $u = 0$,~$v = 0$, and that it lies in the straight
line $u + n v = 0$. Then since on putting $x + h$,~$y + k$
for $x$ and $y$, we have
\begin{flalign*}
& & (v w)^1 = &(v w) + (a_4 v + a_3 w) h + (b_4 v + b_3 w) k && \\
&\text{and } \therefore
& &\mathrm{A} = \mathrm{w} (a_3 h + b_3 k) &&\\
& & &\mathrm{B} = \mathrm{w} (a_2 h + b_2 k)&&\\
& & &\mathrm{C} = 0 &&
\end{flalign*}
Where w is the value of $w$, for the values of $x$ and $y$
determined by $u = 0$, $v = 0$.

Moreover from the equation $u + n v = 0$
\[
a_2 h + b_2 k + n (a_3 h + b_3 k) = 0
\]
\begin{flalign*}
&\text{Hence} & \mathrm{B} + n \mathrm{A} & = 0 &&\\
& \text{and} & \mathrm{A} \lambda + \mu \mathrm{B}  & = 0 &&\\
& \text{and} \therefore &  \lambda - n \mu & = 0 &&\\
& \therefore & u (\mathrm{L}u - \mathrm{K})- n v & (\mathrm{M} v - \mathrm{K}) = 0 &&
\end{flalign*}
is the ultimate state of equation (\textit{b}). This latter is
therefore the locus of the centres of all conic sections
which can be drawn through two given points
$u~w$,~$v~w$, and touching a given straight line
$u + n v = 0$ in a given point $u~v$.
\addcontentsline{toc}{section}{Locus of centres of all conic sections passing
through three given points, and touching a given straight line}

\bigskip
$3^\textrm{o}$ Let $\lambda$,~$\mu$,~$\nu$ be connected by the equation
\[
(\mathrm{A} \lambda)^\frac{1}{2} + (\mathrm{B} \mu)^\frac{1}{2} +
(\mathrm{C} \nu)^\frac{1}{2} = 0 \tag{$\beta$}
\]
and in conformity with this condition let us seek
the \textit{envelope} of (1):

Diff. (1) and ($\beta$) relatively to $\lambda$,~$\mu$,~$\nu$, we have
\[
  vw\;d\lambda + uw\;d\mu + uv\;d\nu = 0
\]
or
\[
\frac{1}{u}\;d \lambda + \frac{1}{v}\;d \mu + \frac{1}{w}\;d \nu = 0
\]
\smallskip
\noindent\[
  \frac{\mathrm{A}^\frac{1}{2}}{\lambda^\frac{1}{2}}\;\,d\lambda +
  \frac{\mathrm{B}^\frac{1}{2}}{    \mu^\frac{1}{2}}\;\,d\mu +
  \frac{\mathrm{C}^\frac{1}{2}}{    \nu^\frac{1}{2}}\;\,d\nu = 0
\]

\bigskip
\noindent Hence
$\lambda^\frac{1}{2} = k \mathrm{A}^\frac{1}{2} u$,~
$    \mu^\frac{1}{2} = k \mathrm{B}^\frac{1}{2} v$,~
$    \nu^\frac{1}{2} = k \mathrm{C}^\frac{1}{2} w$
putting which in ($\beta$) we have
\[
\mathrm{A}u + \mathrm{B}v + \mathrm{C}w = 0
\]
for the envelope required. We may therefore
consider ($\beta$) as the condition that the curve (1)
passing through three given points may also touch
a given straight line $t = 0$, for we have only to determine
A,~B,~and~C, so that
\[
\mathrm{A}u + \mathrm{B}v + \mathrm{C}w = t
\]
identically. Substituting the values of $\lambda$,~$\mu$,~$\nu$, in ($\beta$) we
have for the locus of the centres of a system of
conic sections passing through three given points
and touching a given straight line,
\[
  \{ \mathrm{A}u (\mathrm{L}u - \mathrm{K}) \}^\frac{1}{2}
+ \{ \mathrm{B}v (\mathrm{M}v - \mathrm{K}) \}^\frac{1}{2}
+ \{ \mathrm{C}w (\mathrm{N}w - \mathrm{K}) \}^\frac{1}{2} = 0
\tag{\textit{c}}
\]
which being rationalized will be found to be of the
fourth order.
\addcontentsline{toc}{section}{Equation to a conic section touching three given straight lines}

\bigskip
$4^\textrm{o}$ Let $u = 0$,~$v = 0$,~$w = 0$ be the equations to
three given straight lines,
\[
(\lambda u )^{\frac{1}{2}} + (\mu v)^{\frac{1}{2}} + (\nu w)^{\frac{1}{2}} = 0 \tag{2}
\]
will be the equation necessary to a conic section
touching each of those lines.

For the equation in a rational form is
\[
\lambda^2 u^2 + \mu^2 v^2 + \nu^2 w^2 = 2 \{\lambda \mu u v + \lambda \nu u w + \mu \nu v w\}
\]
Make $w = 0$ and it reduces to
\[
(\lambda u - \mu v)^2 = 0
\]
and hence the points common to (2) and $w = 0$
will be determined by the simultaneous equations
$w = 0$ and $\lambda u - \mu v = 0$. But these being linear,
determine only one point. Hence $w = 0$ is a tangent
to (5). Similarly $u = 0$, $v = 0$ are tangents.
\addcontentsline{toc}{section}{Equation to a conic section touching four given straight lines}

\bigskip
$5^\textrm{o}$ Let $\lambda$,~$\mu$,~$\nu$, be connected by the equation
\[
\frac{\lambda}{\mathrm{A}}+\frac{\mu}{\mathrm{B}}+\frac{\nu}{\mathrm{C}}=0\tag{$\gamma$}
\]
where A,~B,~C are fixed constants, and consistently
with this condition let us seek the envelope of (2).

Differentiating (2) and ($\gamma$) with respect to $\lambda$,~$\mu$,~$\nu$,
\[
\lambda^{-\frac{1}{2}} u^{\frac{1}{2}}\; d\lambda + \mu^{-\frac{1}{2}} v^{\frac{1}{2}}\; d\mu + \nu^{-\frac{1}{2}} w^{\frac{1}{2}}\; d\nu = 0
\]
\[
\frac{d \lambda}{\mathrm{A}}+\frac{d\mu}{\mathrm{B}}+\frac{d\nu}{\mathrm{C}} = 0
\]
\begin{flalign*}
&\indent \text{Hence }
& & \frac{k}{\mathrm{A}} = \lambda^{-\frac{1}{2}} u^{\frac{1}{2}}
\qquad \text{or} \qquad k^2 \lambda = \mathrm{A}^2u && &&
\end{flalign*}
$k$ being an arbitrary factor.
\begin{flalign*}
&\indent \text{Also} & &  k^2 \mu=\mathrm{B}^2v \qquad k^2\nu=\mathrm{C}^2w&& &&
\end{flalign*}
putting which in (2) we have
\[
\mathrm{A}u+\mathrm{B}v+\mathrm{C}w=0
\]
for the envelope required, and which being linear
represents a straight line.

Hence, if $t = 0$ be the equation to a fourth
straight line, and $A$, $B$, $C$ be determined by
making
\[
\mathrm{A}u+\mathrm{B}v+\mathrm{C}w\quad\text{identical with} \quad t
\]
equation (2) subject to the condition ($\gamma$) will represent
all conic sections capable of simultaneously
touching four given straight lines,
\[
t = 0, \quad u = 0, \quad v = 0, \quad w = 0.
\]

Expanding the equation (2) into its rational integral
form, and differentiating  with respect to
$x$~and~$y$, \medskip and putting the differential co-efficients
$\dfrac{d(2)}{dx}$,~$\dfrac{d(2)}{dy}$ \medskip separately $= 0$, we get two equations
for the co-ordinates of the centre. Those equations
may be exhibited thus:
\[
\frac{\lambda (a_2 b_4 - a_4 b_2) + \mu (a_4 b_3 - a_3 b_4)}{w} \qquad \qquad \qquad \qquad
\]
\[
=  \frac{\nu (a_4 b_3 - a_3 b_4) + \lambda (a_3 b_2 - a_2 b_3)}{v}
\]
\[
=  \frac{\mu (a_3 b_2 - a_2 b_3) + \nu (a_2 b_4 - a_4 b_2)}{u}
\]
\[
\text{or,} \qquad \frac{\lambda\mathrm{M}+\mu\mathrm{L}}{w}=\frac{\nu\mathrm{L}+
\lambda\mathrm{N}}{v}=\frac{\mu\mathrm{N}+\nu\mathrm{M}}{u} \smallskip
\]
where L,~M,~N are determined as before by making
\[
\mathrm{L}u+\mathrm{M}v+\mathrm{N}w=2\mathrm{K}\quad\text{identically.}
\]
The preceding equations give
\begin{align*}
\lambda & =\mathrm{L}(\mathrm{L}u-\mathrm{K}) \\
\mu & =\mathrm{M}(\mathrm{M}v-\mathrm{K}) \\
\nu & =\mathrm{N}(\mathrm{N}v-\mathrm{K})
\end{align*}
and putting these in the condition ($\gamma$) we have
\[
0=\frac{\mathrm{L}(\mathrm{L}u-\mathrm{K})}{\mathrm{A}}+\frac{\mathrm{M}(\mathrm{M}v-\mathrm{K})}{\mathrm{B}}+\frac{\mathrm{N}(\mathrm{N}w-\mathrm{K})}{\mathrm{C}}
\]
for the locus of centres which being linear in
$u$,~$v$,~$w$, will be linear in $x$~and~$y$, and therefore
represents a straight line.
\addcontentsline{toc}{section}{Locus of centres of all conic sections touching four given
straight lines}

\bigskip
$6^\textrm{o}$. Resuming again the equation (2), and making
$\lambda$,~$\mu$,~$\nu$, subject to the condition
\[
(\mathrm{A}\lambda)^\frac{1}{2}+(\mathrm{B}\mu)^\frac{1}{2}+(\mathrm{C}\nu)^\frac{1}{2}=0,
\]
which will restrict the curve (2) to pass through a
given point, A,~B,~C being the values of $u$,~$v$,~and~$w$,
for that point. Putting in the values of $\lambda$,~$\mu$,~$\nu$, determined
above, we have
\[
\{\mathrm{AL}(\mathrm{L}u-\mathrm{K})\}^\frac{1}{2}+
\{\mathrm{BM(M}v-\mathrm{K})\}^\frac{1}{2}+
\{\mathrm{CN}(\mathrm{N}w-\mathrm{K})\}^\frac{1}{2}=0
\]
for the locus of centres.

\addcontentsline{toc}{section}{Locus of centres of all conic sections
touching three given straight lines, and passing through a given point,
and very curious property deduced as a corollary}
Hence the locus of the centres of all conic sections
which touch three given straight lines and
pass through a given point is also a conic section.\medskip

\textsc{Cor.} From the form of the equation this locus
touches the lines \smallskip $u = \dfrac{\textrm{K}}{\textrm{L}}$, $v = \dfrac{\textrm{K}}{\textrm{M}}$, $w = \dfrac{\textrm{K}}{\textrm{N}}$, \smallskip which are
parallel to the given lines and at the same distances
from them respectively wherever the \textit{given} point
may be situated, L,~M,~N,~K, being independent
of A,~B,~C. In fact, it is easy to demonstrate that
they are the three straight lines joining the points
of bisection of the sides of the triangle formed by
$u$,~$v$,~$w$, and hence the following theorem.

If a system of conic sections be described to
pass through a given point and to touch the sides
of a given triangle, the locus of their centres will
be another conic section touching the sides of the
co-polar triangle which is formed by the lines joining
the points of bisection of the sides of the
former.
\addcontentsline{toc}{section}{Equation to a conic section touching two given straight lines,
and passing through two given points and locus of centres}

\bigskip
$7^\textrm{o}$. We now proceed to the case of a conic section
touching two given straight lines, and passing
through two given points. Let $u = 0$, $v = 0$, be
the equations of the two lines touched, and $w = 0$
the equation of the line passing through the two
given points. Then taking the equation
\[
(\lambda u)^\frac{1}{2} + (\mu v)^\frac{1}{2} + (\nu w + 1)^\frac{1}{2} = 0
\]
we know by the preceding that this represents a
conic section touching \smallskip $u = 0$, $v = 0$, and $w + \dfrac{1}{\nu}= 0$. \smallskip
Let $\alpha$,~$\alpha'$ be the values of $u$ at the given points, and
$\beta$,~$\beta'$ those of $v$, the values of $w$ being zero for each,
then the equations for finding $\lambda$~and~$\mu$ will be
\begin{align*}
  &(\lambda \alpha)^\frac{1}{2}\; + (\mu \beta)^\frac{1}{2}\; + 1 = 0 \\
  &(\lambda \alpha')^\frac{1}{2} + (\mu \beta')^\frac{1}{2} + 1 = 0;
\end{align*}
Let A~and~B be the values of $\lambda$~and~$\mu$ deduced from
these, and we have for the equation of the conic
section
\[
(\mathrm{A}u)^\frac{1}{2}+(\mathrm{B}v)^\frac{1}{2}+(\nu w+1)^\frac{1}{2}=0
\]
in which $\nu$ is the only arbitrary constant.

Differentiating this equation when expanded into
its rational form with respect to $x$ and $y$, we have
two equations respectively equivalent to
\[
\nu=-\mathrm{N}\cdot\frac{\mathrm{A}u-\mathrm{B}v}{\mathrm{L}u-\mathrm{M}v}
\]
\[
\nu w+1=\frac{\mathrm{AM}+\mathrm{BL}}{\mathrm{AM}-\mathrm{BL}}(\mathrm{A}u-\mathrm{B}v)
\]
determining L,~M,~N, as before, by making
\[
\mathrm{L}u+\mathrm{M}v+\mathrm{N}w=2\mathrm{K}\quad \text{identically.}
\]
Hence eliminating $\nu$, there arises
\[
2 \{\mathrm{B}\mathrm{L}(\mathrm{L}u-\mathrm{K})-\mathrm{A}\mathrm{M}(\mathrm{M}v-
\mathrm{K})\}(\mathrm{A}u-\mathrm{B}v)
=(\mathrm{A}\mathrm{M}-\mathrm{B}\mathrm{L})(\mathrm{L}u-\mathrm{M}v)
\]
for the locus of the centres.

This is also a curve of the second order, and the
values of A and B are
\begin{align*}
&\mathrm{A}=\left\{\frac{\beta^\frac{1}{2}-\beta^{'\frac{1}{2}}}
{(\alpha\beta^{'})^\frac{1}{2}-(\alpha^{'}\beta)^\frac{1}{2}}\right\}^2
&\mathrm{B}=\left\{\frac{\alpha^{'\frac{1}{2}}-\alpha^\frac{1}{2}}
{(\alpha\beta{'})^\frac{1}{2}-(\alpha{'}\beta)^\frac{1}{2}}\right\}^2
\end{align*}
This demonstration assumes that it is possible to
draw a tangent to each of the system of curves
parallel to $w = 0$. But in case the given points are
in opposite vertical angles of the given straight
lines, and the curves therefore hyperbolas, this will
not be possible, and accordingly in such case the
values of A and B become imaginary, for in this
case $\alpha$,~$\alpha'$, as also $\beta$,~$\beta'$ have different signs. The
following method is free from this and every objection,
and is perfectly general.
\addcontentsline{toc}{section}{Another mode of investigating preceding}

\bigskip
$8^\textrm{o}$. Let $u = 0$, and $v = 0$, as before, be the equations
to the tangents, $w = 0$ the straight line joining
the two given points, and $w' = a'_4 x + b'_4 y$, $a'_4$
and $b'_4$ being determined as follows:
\begin{align*}
w'_{\alpha \beta} &= a'_4 \alpha + b'_4 \beta =(u v)^\frac{1}{2}_{\alpha \beta}\\
w'_{\alpha' \beta'} &= a'_4 \alpha' + b'_4 \beta' =(u v)^\frac{1}{2}_{\alpha' \beta'}
\end{align*}
$\alpha~\beta$; $\alpha'~\beta'$ being co-ordinates of the given points.
\begin{flalign*}
&\text{Then} & &(w + m w')^2 = m^2 u v &&
\end{flalign*}
is the equation to the system in which $m$ is arbitrary.
For $u = 0$, or $v = 0$, each give $w + m w' = 0$,
and therefore $u$ and $v$ each touch the curve, and
$w + m w' = 0$ is the equation to the line joining
their points of contact. Again, by the preceding
determination of $w'$, we have for $x = \alpha$, $y = \beta$,
$w = 0$, and $m^2 w'^{2}_{\alpha \beta} = m^2 (u v)_{\alpha \beta}$
and similar for $\alpha'~\beta'$, and hence $m$ remains arbitrary.
Differentiating the equation
\[
(w + m w')^2 = m^2 u v
\]
with respect to $x$ and $y$,
\begin{flalign*}
&& m^2 u v &= (w + m w')^2 && &&\\
&& m^2 (a_2 v + a_3 u) &= 2 (w + m w') (a_4 + m a'_4) && &&\\
&& m^2 (b_2 v + b_3 u) &= 2 (w + m w') (b_4 + m b'_4) && &&\\
&& m^2\{\mathrm{L}u-\mathrm{M}v\} & =2m(w+mw')(a'_4b_4-a_4b'_4) && &&\\
&& m\{\mathrm{L}u-\mathrm{M}v\} & =2\mathrm{Q}(w+mw') && &&\\
&& m^2\{\mathrm{L}'u-\mathrm{M}'v\} & =-2\mathrm{Q}(w+mw') && &&\\
&& \therefore \quad m & =-\frac{\mathrm{L}u-\mathrm{M}v}{\mathrm{L}'u-\mathrm{M}'v} && &&\\
&& \therefore \quad w+mw' & =w-\frac{\mathrm{L}u-\mathrm{M}v}{\mathrm{L}'u-\mathrm{M}'v}w' && &&\\
&& \frac{(\mathrm{L}u-\mathrm{M}v)^2}{\mathrm{L}'u-\mathrm{M}'v}+2\mathrm{Q} & 
\left\{w-\frac{\mathrm{L}u-\mathrm{M}v}{\mathrm{L}'u-\mathrm{M}'v}w'\right\} =0; && &&\\
& \text{or,}
& (\mathrm{L}u-\mathrm{M}v)^2+2\mathrm{Q}\{(\mathrm{L}'u- & \mathrm{M}'v)w-(\mathrm{L}u-\mathrm{M}v)w'\}=0, && &&
\end{flalign*}
which is an equation of the second order.

Now let $u_{\alpha \beta}$,~$v_{\alpha \beta}$ be both positive, and
$u_{\alpha' \beta'}$,~$v_{\alpha' \beta'}$
both negative, and therefore the given points in opposite
vertical angles of the straight lines $u = 0$ and
$v = 0$. Then $a'_4$ and $b'_4$ will both be real quantities,
and $\therefore$ also Q,~L$'$,~M$'$, and $w'$. Also if $u_{\alpha \beta}$,
$v_{\alpha \beta}$ have different signs, as also $u_{\alpha' \beta'}$, $v_{\alpha' \beta'}$ then
$a'_4$,$b'_4$, Q, L$'$, M$'$, and $w'$ will be of the form A$\sqrt{-1}$
and the above equation equally real.

\bigskip
$9^\textrm{o}$. We have now discussed the several cases of
the general problem, whose enunciation is as follows:

Of four straight lines and four points let any four
be given, and draw a system of conic sections
passing through the given points, and touching the
given lines, to investigate the locus of the centres.

We have shown that in every case except two
the locus is a conic section. The two exceptions
are, first, when there are three given
points and a given straight line, in which case the
locus is
\[
\{\mathrm{A}u(\mathrm{L}u-\mathrm{K})\}^\frac{1}{2}+\{\mathrm{B}v(\mathrm{M}v-\mathrm{K})\}^\frac{1}{2}+
\{\mathrm{C}w(\mathrm{N}w-\mathrm{K})\}^\frac{1}{2}=0
\]
which, being rationalized, is of the fourth order.

But some doubt may exist as to whether such
equation may not be decomposable into two quadratic
factors, and thus represent two conic sections.
That such cannot hold generally will best appear
from the discussion of a particular case.

The other case of exception is when the data are
four straight lines, the locus then being a straight
line; but since a straight line may be included
amongst the conic sections, we may say that there
is but one case of exception.

The particular case we propose to investigate is
the following.

\addcontentsline{toc}{section}{Investigation of a particular case of conic sections passing through
three given points, and touching a given straight line; locus of
centres a curve of third order, the hyperbolic cissoid}
Through one of the angular points of a rhombus
draw a straight line parallel to a diagonal, and
let a system of conic sections be drawn, each touching
the parallel to the diagonal, and also passing
through the three remaining angular points of the
rhombus, to investigate the locus of centres.

%[Illustration]
\begin{figure}[htbp]
\begin{center}
\includegraphics*[width=10cm]{images/fig5}
\end{center}
\end{figure}

Let $\mathrm{AO} = \mathrm{OD} = 1$, $\tan \mathrm{BAO} = \tan \mathrm{CAO} = m$.
Taking the origin at O the equations are
\begin{flalign*}
&& && &u = y - m (x + 1) = 0 & &\text{for AB} && && \\
&& && &v = y + m (x + 1) = 0 & &\text{for AC} && && \\
&& && &w = x = 0 & &\text{for BC} && && \\
&& && \textrm{A} u + &\textrm{B} v + \textrm{C} w = x - 1 = 0 & &\text{for KD.} && &&
\end{flalign*}
To determine A,~B,~C, we have therefore
\begin{flalign*}
&& \textrm{A} \bigl(y - m (x + 1)\bigr) + \textrm{B} \bigl(y &+ m (x + 1)\bigr) + \textrm{C} x = x - 1 && && &&\\
&\text{identically;} & \therefore \quad \mathrm{A} + \mathrm{B} & = 0 && && &&\\
&& - m \textrm{A} + m \textrm{B} + \textrm{C} & = 1 && && &&\\
&& - m \textrm{A} + m \textrm{B} & = -1 && && &&\\
&&  \therefore \quad \textrm{C} = 2, \quad \textrm{A} = \frac{1}{2 m}, & \quad  \textrm{B} = -\frac{1}{2 m}. && && &&
\end{flalign*}
Also for finding L,~M,~N
\begin{align*}
\mathrm{L}\biggl(y-m(x+1)\biggr)+&\mathrm{M}\biggl(y+m(x+1)\biggr)+\mathrm{N}x=2\mathrm{K}\\
&\mathrm{L}+\mathrm{M}=0\\
-m\mathrm{L}&+m\mathrm{M}+\mathrm{N}=0\\
-m\mathrm{L}&+m\mathrm{M}=2\mathrm{K}\\
\therefore\quad\mathrm{N}=-2\mathrm{K},\quad&\mathrm{L}=-\frac{\mathrm{K}}{m},\quad\mathrm{M}=\frac{\mathrm{K}}{m}
\end{align*}
by the substitution of which in (c) we have for the
locus of the centres,
\[
\bigl\{u (u + m) \bigr\}^\frac{1}{2} + \bigl\{v (v - m) \bigr\}^\frac{1}{2}
+ \bigl\{4 m^2 w (2 w + 1) \bigr\}^\frac{1}{2} = 0,
\]
or
\[
  \biggl\{ \{y - m (x + 1) \} \{y - m x \} \biggr\}^\frac{1}{2} +
  \biggl\{ \{y + m (x + 1) \} \{y + m x \} \biggr\}^\frac{1}{2}
  + 2 m \bigl\{x (2 x + 1)\bigr\}^\frac{1}{2} = 0
\]
and this equation rationalized and reduced gives
\[
(2 x - 1) (2 x + 1) y^2 = 4 m^2 x^3 (2 x + 1)
\]
which resolves into the two
\begin{flalign*}
&& &2x + 1 = 0 &&\\
&\text{and}&
&y^2 = \frac{4 m^2 x^3}{2x - 1}&&
\end{flalign*}
the first of these, since
\[
\nu = w (\mathrm{N} w - \mathrm{K}) = -\mathrm{K} x (2 x + 1)
\]
requires $\nu = 0$, which would reduce the equation
of the system to $\lambda v w + \mu u w = 0$, or
$w = 0$, $\lambda v + \mu u = 0$, representing only two straight
lines.  \smallskip Hence $2 x + 1 = 0$
must be rejected, and therefore $y^2 = \dfrac{4 m^2 x^3}{2x - 1}$ \smallskip
is the required locus.

Now this curve is essentially one of the third
order, and generated from the hyperbola in the same
manner as the cissoid of Diocles is from the circle.
This we proceed to demonstrate.

\addcontentsline{toc}{section}{Genesis and tracing of the hyperbolic cissoid}
If $x$ be measured in the contrary direction OA,
the equation may be written
\[
y^2=\frac{4m^2x^3}{2x+1}
\]

%[Illustration.]
\begin{figure}[htb]
\begin{center}
\includegraphics*[width=10cm]{images/fig6}
\end{center}
\end{figure}

Let a hyperbola be described of which the semi
axes are, real $=\frac{1}{4}$ , \smallskip imag.~$=\dfrac{m\sqrt{2}}{4}$, \smallskip and through the vertex
A draw any line NAP\@. Draw NQ and make
CM~=~CQ\@. Also draw the ordinate RM, cutting
NAP in P, then P will be a point in the curve.

For let $y=\alpha x$ be equation to NAP, A being
origin, equation to hyperbola $y^2 = m^2 x + 2m^2 x^2$,
$\;\therefore\;$ for intersection N, $\alpha^2 x = m^2 + 2 m^2 x$
\[
x=-\frac{m^2}{2m^2-\alpha^2}; \quad \therefore \quad \mathrm{AQ}=\frac{m^2}{2m^2-\alpha^2}
\]
\begin{flalign*}
&\text{Also}& \mathrm{BQ } = \mathrm{AP} &= x,
\quad \frac{m^2}{2m^2-\alpha^2} -x = \mathrm{AB} = \frac{1}{2} &&\\
&& &\frac{2m^2}{2m^2-\alpha^2}=2x+1; &&\\
&& \therefore \quad &2 m^2 - \alpha^2 = \frac{2 m^2}{2 x + 1} &&\\
&& &\alpha^2 = \frac{4 m^2 x}{2 x + 1} = \frac{y^2}{x^2} &&\\
&& &\therefore \quad  y^2 = \frac{4 m^2 x^3}{2x + 1}; &&
\end{flalign*}
Hence the locus of P is the curve in question.\medskip

\noindent To find the asymptotes.\smallskip

Taking the equation
\[
y^2 = \frac{4 m^2 x^3}{2x - 1}
\]
\[
  y = \pm 2 m x^{\frac{3}{2}} (2 x)^{-\frac{1}{2}} \left\{1 -
  \frac{1}{2 x} \right\}^{-\frac{1}{2}}
\]
\[
  = \pm \sqrt{2} m x \left\{1 + \frac{1}{4 x} + \frac{1}{2} \cdot
  \frac{3}{4} \cdot \frac{1}{4 x^2} \ \ \&c. \right\}
\]
so that for $x$ infinite we have
\[
y = \pm \sqrt{2} m \left(x + \frac{1}{4} \right);
\]
Moreover since $x = \frac{1}{2}$ makes $y$ and $\dfrac{dy}{dx}$ infinite,
the equation
\[
2 x - 1 = 0
\]
gives another asymptote.

These asymptotes being drawn, the curve will be
found to consist of three distinct branches, as in the
figure.
\addcontentsline{toc}{section}{Equation to a conic section touching three given
straight lines, and also the conic section passing through the mutual\newline intersections
of the straight lines and locus of centres}

\bigskip
$10^\textrm{o}$. What we have hitherto exhibited seems to
be far from being the full extent of applicability
of this method of investigation, as the following
will show.
\begin{flalign*}
&\indent\text{Let} & &\mathrm{A} v w + \mathrm{B} u w + \mathrm{C} u v = 0& \tag{3}
\end{flalign*}
be a \textit{fixed} conic section passing through the intersections
of $u = 0$, $v = 0$, $w = 0$. It is required to find
a system of conic sections, each of which shall touch
the lines $u = 0$, $v = 0$, $w = 0$, and also the curve (3).

\begin{flalign*}
&\indent\text{Let} & &(\lambda u)^\frac{1}{2} + (\mu v)^\frac{1}{2} +
(\nu w)^\frac{1}{2} = 0& \tag{4}
\end{flalign*}
be the equation to any curve of the system.

This already touches $u$,~$v$,~$w$, and if we assume
\[
  (\mathrm{A} \lambda)^\frac{1}{3} + (\mathrm{B} \mu)^\frac{1}{3} +
  (\mathrm{C} \nu)^\frac{1}{3} = 0 \tag{$\delta$}
\]
and investigate the envelope of (4) we find it to be
no other than the equation (3).

Hence the equation (4), in which $\lambda$,~$\mu$,~$\nu$, are subject
to the condition ($\delta$), represents the required
system.

Hence the locus of the centres of the system is
\[
\{\mathrm{A}\mathrm{L}(\mathrm{L}u-\mathrm{K})\}^\frac{1}{3}+\{\mathrm{B}\mathrm{M}(\mathrm{M}v-\mathrm{K})\}^\frac{1}{3}
+\{\mathrm{C}\mathrm{N}(\mathrm{N}w-\mathrm{K})\}^\frac{1}{3}=0.
\]
\addcontentsline{toc}{section}{Equation to a conic section passing through the mutual\newline intersections
of three tangents to another conic section, and also touching the
latter and locus of centres}

\bigskip
$11^\textrm{o}$. Now let
\[
(\mathrm{A} u)^\frac{1}{2} + (\mathrm{B} v)^\frac{1}{2} +
(\mathrm{C} w)^\frac{1}{2} = 0 \tag{5}
\]
be a fixed conic section touching $u = 0$,~$v = 0$,~$w = 0$,
and let it be required to find a system of conic sections,
each passing through the mutual intersections
of $u = 0$,~$v = 0$,~$w = 0$, and also touching (5).
\begin{flalign*}
&\text{Let}& &\lambda v w + \mu u w + \nu u v = 0 && \tag{6}
\end{flalign*}
be the equation to each curve of the system, and
suppose $\lambda$,~$\mu$,~$\nu$ connected by ($\delta$) as before.

On investigating the envelope of (6) we find it
to be no other than (5). Hence (6), subject to
condition ($\delta$), represents the system required.

The locus of the centres in this case will be
\[
\{\mathrm{A} u (\mathrm{L} u -\mathrm{K})\}^\frac{1}{3} + \{\mathrm{B} v (\mathrm{M} v - \mathrm{K})\}^\frac{1}{3}
+ \{\mathrm{C} w (\mathrm{N} w - K)\}^\frac{1}{3} = 0.
\]

This curve is double the dimensions of that in
the preceding case, and each result assures us that
were we to find the solution of the following,
``To find the locus of the centres of systems of
conic sections, each of which touches four given
conic sections,'' we should have an algebraical
curve of very high dimensions, and not in general
resolvable into factors, each representing a curve of
the second order.
\addcontentsline{toc}{section}{Solution to a problem in Mr.\ Coombe's
Smith's prize paper\newline
for 1846}

\bigskip
I will conclude this chapter by applying my
method to solve a theorem proposed by Mr. Coombe
in his Smith's Prize Paper of the present year.

The theorem is, ``If a conic section be inscribed
in a quadrilateral, the lines joining the points of
contact of opposite sides, each pass through the
intersection of the diagonals.''

Let $u=0$,~$v=0$,~$w=0$,~$t=0$, be the equations
to the sides of the quadrilateral;

Then determining A, B, C, by making
\[
 \mathrm{A}u+\mathrm{B}v+\mathrm{C}w=t, \qquad \textit{identically} \tag{1}
\]
And subjecting $\lambda$,~$\mu$,~$\nu$, to the condition
\[
\frac{\lambda}{\mathrm{A}}+\frac{\mu}{\mathrm{B}}+\frac{\nu}{\mathrm{C}}=0 \tag{2}
\]
\begin{flalign*}
&\text{we have} & &(\lambda u)^\frac{1}{2}+(\mu v)^\frac{1}{2}+(\nu w)^\frac{1}{2}=0& \tag{3}
\end{flalign*}
for the inscribed conic section.

But equation (3) may be put in the form
\[
4\mu\nu v w = (\lambda u - \mu v - \nu w)^2
\]
\begin{flalign*}
&\text{so that} &&\qquad\lambda u - \mu v - \nu w=0 && &&
\end{flalign*}
is the equation to the line joining the points of
contact of $v$ and $w$.

From (1) we have
\[
\mathrm{A}u+\mathrm{B}v\quad\text{identical with}\quad t-\mathrm{C}w,
\]
so that either of these equated to zero will represent
the diagonal DB, and similarly
$\mathrm{A}u+\mathrm{C}w=0$, or $t-\mathrm{B}v=0$ will represent the
diagonal AC.

\begin{flalign*}
&\text{But from}&  &\mathrm{A}u+\mathrm{B}v=0& &&\\
&& &\mathrm{A}u+\mathrm{C}w=0 & &&\\
&\text{and} & &\frac{\lambda}{\mathrm{A}} + \frac{\mu}{\mathrm{B}} + \frac{\nu}{\mathrm{C}} = 0& &&
\end{flalign*}

\noindent Eliminating A,~B,~C, we obtain
\[
\lambda u - \mu v - \nu w=0
\]

Hence this line passes through the intersection of
the two diagonals. But this has been shown to be
the line joining the points of contact of the opposite
sides $v$,~$w$; such line of contact therefore passes
through the intersection of diagonals. Similarly, the
other line of contact also passes through the intersection
of diagonals.




\begin{center}
\chapter{CHAPTER III.}
\end{center}


I have applied analysis after the method followed
in the last Chapter to the solution of a vast
number of very general theorems, and always with
complete success. I have also extended it to three
dimensions, and have discovered many remarkable
properties and relations hitherto unknown, and
have also obtained very concise and elegant demonstrations
of known theorems. It is intended
in this Chapter to instance a few of them.

\addcontentsline{toc}{section}{Equation to a surface of second order, touching three planes in points
situated in a fourth plane}
Let $u$,~$v$,~$w$,~$t$ be linear functions of $x$,~$y$,~$z$, and
let the planes
\[
u=0, \quad v=0, \quad w=0,
\]
be supposed to touch a surface of the second order
in points situated in the plane $t=0$; then the
equation to that surface will be as follows:
\[
\mathrm{A}^2u^2+\mathrm{B}^2v^2+\mathrm{C}^2w^2 - 2\mathrm{AB}uv - 2\mathrm{AC}uw - 2\mathrm{BC}vw \pm t^2 =0.
\]

For suppose $u=0$, the equation becomes
\[
(\mathrm{B}v-\mathrm{C}w)^2 \pm t^2 =0.
\]

Taking the upper sign this requires
\[
\mathrm{B}v-\mathrm{C}w=0, \; \text{and} \; t=0;
\]
\noindent and taking the lower sign
\[
\mathrm{B}v-\mathrm{C}w+t=0,\; \text{and} \; Bv-Cw-t=0.
\]

In either case the point determined by
\[
u=0, \quad \mathrm{B}v-\mathrm{C}w=0, \quad \text{and} \quad t=0,
\]
will be a point in the surface to which the plane
$u=0$ is tangential. In the former case it will
touch the surface in this point only, in the latter it
will touch in this point and cut in the straight
lines
\[
\mathrm{B}v-\mathrm{C}w+t=0,\quad \text{and} \quad \mathrm{B}v-\mathrm{C}w-t=0.\footnote{
This is the case of the hyperboloid of one sheet. If $t$ be
obliterated there is but one line, and the surface becomes a cone
whose vertex is the common intersection of $u$,~$v$,~$w$.}
\]

In the same way $v=0$, and $w=0$ are also tangent
planes.

Now $\mathrm{B}v-\mathrm{C}w=0$ represents a plane through
the common intersection of the planes $v=0$,
$w=0$. Similarly $\mathrm{A}u-\mathrm{B}v=0$ represents a
plane through the common intersection of $u=0$,
$v=0$; and $\mathrm{C}w-\mathrm{A}u=0$ one through the intersection
of $w=0$, $u=0$.

Whence, since $\mathrm{C}w-\mathrm{A}u=0$ is a consequence
of $\mathrm{A}u-\mathrm{B}v=0$, and $\mathrm{B}v-\mathrm{C}w=0$, we may
assert the following theorem:

\addcontentsline{toc}{section}{Theorems deduced from the above}
If a surface of the second order be tangential
to three planes, the planes passing through the
mutual intersections of every two of them and the
point of contact of the third tangent plane, will
intersect in the same straight line.

Again, let straight lines be drawn from the point
of mutual intersection of $u=0$,~$v=0$,~$w=0$, one
in each plane, and let two surfaces of the second
order touch the three planes $u=0$,~$v=0$,~$w=0$,
in points respectively situated in those straight
lines, then the equations to the two surfaces differing
only in the value of $t$, we have at their
points of intersection
\[
t^2-t'^2=0,
\]
\[
 \text{or} \quad t-t'=0,  \quad t+t'=0.
\]

Hence if two surfaces of the second order touch
three planes in such a manner that the lines joining
points of contact on each plane all pass through
the point of common intersection of the three
planes, the surfaces (if they intersect at all) intersect
in one plane, or else in two planes.

In M. Chasles' Memoirs on Cones and Spherical
Conics, translated by the Rev. Charles
Graves, F.T.C.D., I find the following remark:

``These theorems might also be demonstrated
by algebraic analysis; but this method, which in
general offers so great advantages, loses them all
in this case, since it often requires very tedious
calculations, and exhibits no connexion between
the different propositions; so that it is only useful
in verifying those which are already known, or
whose truth has been otherwise suggested as probable.''

All who have read M. Chasles' Memoirs must
greatly admire the exquisite ingenuity and generalization
displayed in them; but I think no one
who well understands the use of analysis, and is
capable of applying it to the utmost advantage,
would readily subscribe to the preceding remark.

I would unhesitatingly engage to furnish good
analytical demonstrations of all M. Chasles' theorems,
and as the matter is allied to the subject
of this volume, and would furnish perhaps a
happy illustration to this part of it, I have
adopted the suggestion of a scientific friend to
devote this Chapter to the analytical investigation
of some of Chasles' properties.

\addcontentsline{toc}{section}{Equation to a surface of second order
expressed by means of the equations to the cyclic and metacyclic planes}
Let $u=0$,~$v=0$ be two planes through the
origin, $w=0$ another plane, $r^2 = x^2 + y^2 + z^2$.
\begin{flalign*}
&\text{Then}& &r^2 = m u v + n w & \tag{1}
\end{flalign*}
represents a surface of the second order, of which
$u=0$,~$v=0$, are cyclic planes, and $w = 0$ may
be called the \textit{metacyclic} plane.

For $u =$ const.\ reduces equation (1) to that
of a sphere. The plane $u =$ const.\ necessarily
intersects this sphere in a circle. But the surface
and plane intersect in the same curve as the sphere
and plane; therefore $u =$ const.\ intersects the
surface in a circle. That is, all planes parallel
to $u = 0$ intersect the surface in circles, or in
other words $u = 0$ is a cyclic plane.

Similarly $v = 0$ is a cyclic plane.

If the origin be a point on the surface $w$ is homogeneous
in $x$,~$y$,~$z$, and is a tangent plane to the
surface in the origin.

For $w = 0$ reduces (1) to $r^2 = muv$, from
which, on eliminating $z$, we obtain evidently a
homogeneous linear equation in $x$ and $y$, which will
represent either the origin or two straight lines
through the origin. In either case $w =0$ is a
tangent plane.\footnote{This is also evident from the consideration, that when
the constant term in the general equation of the second or
any higher order is zero, the \textit{linear} part of the equation represents
a plane touching the surface represented by the whole
equation, in the origin.}

Moreover, $u$,~$v$, and $w$ are proportional to the
perpendiculars from a point $x$,~$y$,~$z$, upon those
planes respectively, and may be taken equal to
such perpendiculars by the introduction of proper
multipliers. Hence if $\phi$,~$\phi'$,~$\theta$ are the angles which
$r$ makes with $u$,~$v$,~$w$, we have
\[
  \frac{u}{r} = \sin \phi, \quad \frac{v}{r} = \sin \phi',
  \quad \frac{w}{r} = \sin \theta,
\]
by the substitution of which in (1) we obtain the
condition
\[
r \{1 - m \sin \phi \sin \phi'\} = n \sin \theta.
\]

Hence the following theorem.

\addcontentsline{toc}{section}{General theorems of surfaces of second order
in which one of \newline M.~Chasles' conical theorems is included}
If in any surface of the second order a point
be taken at which a tangent plane is drawn, and
if, moreover, a chord whose length is $r$ be drawn
from the assumed point, and $\theta$ be the angle it
makes with the tangent plane, $\phi$ and $\phi'$, the angles
with the cyclic planes, then
\[
r \{1 - m \sin \phi \sin \phi'\} = n \sin \theta,
\]
where $m$ and $n$ are constant.

When $n = 0$, the surface becomes a cone, origin
at vertex, and $\sin \phi \sin \phi' =$ const., which is one of
Chasles' theorems.

When the chord is in either cyclic plane,
\[
r = n \sin \theta.
\]

Whence if $\delta$ be the diameter of one of the circular
sections through the origin, and $\eta$ the angle
of inclination of its plane to the tangent plane,
\[
\delta = n \sin \eta, \quad \therefore \quad n = \frac{\delta}{\sin \eta}
\]
This therefore determines $n$, which will in general
be different for different points on the surface.
Also $m$ will be the same for all points on the
surface, because if the origin be changed, the axes
remaining parallel to themselves, the terms of
the second order remain the same.

\addcontentsline{toc}{section}{Determination of constants}
In order therefore to determine $m$, suppose the
origin changed to the extremity of one of the
principal axes of the solid, the greatest or least
(not the mean). Let A be the length of such
axis, $\varepsilon$ the angle it makes with either cyclic plane,
D the diameter of a circular section through the
extremity of such axis, then since
\[
\sin \theta = 1, \quad \mathrm{for} \quad \theta = 90^\circ,
\quad \mathrm{and} \quad n = \frac{\mathrm{D}}{\cos \varepsilon},
\]
we have
\[
\mathrm{A} \{1 - m\sin^2 \varepsilon\} = \frac{\mathrm{D}}{\cos \varepsilon};
\quad \therefore \quad m = \frac{\mathrm{A} - \mathrm{D} \sec \varepsilon}{\mathrm{A} \sin^2 \varepsilon}.
\]

The equation of the surface therefore becomes
\[
r^2 = \frac{\mathrm{A}-\mathrm{D}\sec\varepsilon}{\mathrm{A} \sin^2 \varepsilon} u v +
\frac{\delta}{\sin \eta} w.
\]

From this the reader will be able to deduce
many singularly beautiful properties of great
generality.

I will here instance one or two of them:

\addcontentsline{toc}{section}{Curve of intersection of two concentric
surfaces having same cyclic planes}
If two intersecting concentric surfaces of the
second order have the same cyclic planes, they
will intersect in a spherical conic or curve such
that the product of the perpendiculars from any
point in it on the cyclic planes is constant.
\begin{flalign*}
&\text{Let}&  &r^2 = m u v + b^2,& &&\\
&& &r^2 = m' u v + b'^2,& &&
\end{flalign*}
be the two surfaces.

Eliminating $u$~$v$, we have
\[
(m' - m) r^2 = m' b^2 - m b'^2,
\]
the equation to a sphere.
\begin{flalign*}
&\text{Also} & &(b'^2 - b^2) r^2 = (m b'^2 - m' b^2) u v, &&
\end{flalign*}
the equation to a cone vertex at origin.
\begin{flalign*}
&\text {Also} &  &(m -m') u v = b'^2 - b^2;&& \\
&& \therefore \quad &uv = \frac{b'^2 - b^2}{m - m'} = \textrm{const.,}&&
\end{flalign*}
but $u$,~$v$ are the perpendiculars from a point $x,$~$y$,~$z$
on the cyclic planes; this product is therefore
constant for all points on the common intersection
of the two surfaces.

Again let $\Delta$ be the diameter of a sphere touching \smallskip
a surface of the second order in two points,
\textit{and intersecting that surface in circular sections},
then $\Delta = \dfrac{\delta}{\sin \eta}$, and the equation of the surface is
\[
r^2 = \frac{\mathrm{A}-\mathrm{D}\sec\varepsilon}{\mathrm{A}\sin^2 \varepsilon} u v + \Delta w,
\]
the origin being in the surface, and $w = 0$ a tangent
plane through the origin.

Let $w = z$, and make $z = 0$, then
\[
\mathrm{r}^2 = \frac{\mathrm{A}-\mathrm{D}\sec\varepsilon}{\mathrm{A} \sin^2 \varepsilon}
\mathrm{u} \mathrm{v},
\]
r,~u,~v being the values of
$r$,~$u$,~$v$ when $z = 0$.

\addcontentsline{toc}{section}{In an hyperboloid of one sheet the product of the
lines of the angles made by either generatrix with the cyclic planes proved to be
constant, and its amount assigned in known quantities}
This is a homogeneous equation of the second
order in $x$ and $y$, and will therefore represent either
the point of contact, or one or two straight lines.
When it represents one straight line the surface
is conical; when two it is the hyperboloid of one
sheet, the two straight lines being generatrices.
Hence the product of the sines of the angles made
by either generatrix with the cyclic planes is constant
and equal to
\[
\left(\frac{\mathrm{u}}{\mathrm{r}} \cdot \frac{\mathrm{v}}{\mathrm{r}} \right) =
\frac{\mathrm{A}\sin^2\varepsilon}{\mathrm{A}-\mathrm{D} \sec \varepsilon}.
\]

We now proceed to a few of the properties
of Cones.
\addcontentsline{toc}{section}{Generation of cones of the second degree,
and their supplementary cones}

\bigskip
\textit{Generation of Cones of the second degree, and
their Supplementary Cones.}

\medskip
General principle.

Let
\[
\mathrm{A}x+\mathrm{B}y+\mathrm{C}z=0,\;\qquad \frac{x}{\mathrm{A}}=\frac{y}{\mathrm{B}}=\frac{z}{\mathrm{C}},
\]
be a moveable plane and straight line perpendicular
thereto. If the condition to which the motion of
the plane or line be subjected be such that
combined or not with the equation
\[
\mathrm{A}^2+\mathrm{B}^2+\mathrm{C}^2=1
\]
it leads to a homogeneous equation of the second
order in A, B, C, then the plane will envelope
a cone of the second degree, and the line will
generate another cone of the second degree supplementary
to the former.
\begin{flalign*}
&\indent \text{For let}&   &f(\mathrm{A},\mathrm{B},\mathrm{C}) = 0,& &&
\end{flalign*}
\noindent be the homogeneous relation above supposed.
Then since $x$, $y$, $z$, in the equations to the straight
line are proportional to A, B, C, we can replace
the latter by the former in the \textit{homogeneous} equation,
and thus have $f(x, y, z) = 0$, for the
surface described by the moveable straight line.

\begin{flalign*}
&\indent\text{Now let}& \frac{x^2}{a^2} & +\frac{y^2}{b^2}-\frac{z^2}{c^2}=0,& &&\\
\intertext{be the equation of the cone so described by the straight line}
&& & \frac{x}{\mathrm{A}}=\frac{y}{\mathrm{B}}=\frac{z}{\mathrm{C}}; & \\[2mm]
&\indent\textrm{Then}
& \frac{\mathrm{A}^2}{a^2} & +\frac{\mathrm{B}^2}{b^2}-\frac{\mathrm{C}^2}{c^2}=0.&
\end{flalign*}

\medskip
To find the surface enveloped by
\[
\mathrm{A}x+\mathrm{B}y+\mathrm{C}z=0,
\]
\begin{flalign*}
&\text{we have}& \frac{\mathrm{A}}{a^2}\; d &\mathrm{A}+\frac{\mathrm{B}}{b^2}\; d\mathrm{B}-
\frac{\mathrm{C}}{c^2}\; d\mathrm{C}=0, & && \\[3mm]
&& x\,d&\mathrm{A}+y\,d\mathrm{B}+z\,d\mathrm{C}=0;& &&\\
&\therefore&
\frac{\mathrm{A}}{a^2}=\lambda x&,\quad
\frac{\mathrm{B}}{b^2}=\lambda y,\quad
-\frac{\mathrm{C}}{c^2}=\lambda z,& && \\[3mm]
&\text{or}&
\mathrm{A}=\lambda a^2x&,\quad
\mathrm{B}=\lambda b^2y,\quad
\mathrm{C}=-\lambda c^2z.& &&
\end{flalign*}

Putting these in
\[
\mathrm{A}x+\mathrm{B}y+\mathrm{C}z=0,
\]
\[
a^2 x^2 + b^2 y^2 - c^2 z^2 = 0,
\]
the equation to the other cone.

This, therefore, establishes the general principle,
and consequently when any property is predicated
of such cones, the only thing necessary
to be done to demonstrate it is to find whether
combined or not with the condition
\[
\mathrm{A}^2+\mathrm{B}^2+\mathrm{C}^2=1,
\]
it leads to a homogeneous result of the second
order in A, B, C.

\addcontentsline{toc}{section}{Analytical proofs of some of M. Chasles' theorems}
For example. ``The sum or the difference of
the angles which each focal line makes with a
side of the cone of the second degree is constant;''
or in other words, if a straight line drawn from the
point of intersection of two given straight lines
makes angles with them whose sum or difference
is constant, the moveable line traces the surface
of a cone.

Let the equations of the given lines be
\[
\frac{x}{a}=\frac{y}{b}=\frac{z}{c},
\]
\[
\frac{x}{a'}=\frac{y}{b'}=\frac{z}{c'},
\]
and of the moveable line
\[
\frac{x}{\mathrm{A}}=\frac{y}{\mathrm{B}}=\frac{z}{\mathrm{C}},
\]
$\theta$ and $\theta'$ the two angles;
\[
\therefore \qquad \theta \pm \theta' = 2 \; \alpha \; \mathrm{const.};
\]
\[
\therefore \qquad \cos(\theta \pm \theta')=\cos 2\alpha,
\]
\[
 \text{or} \qquad \cos^2\theta+cos^2\theta'-2\cos 2\alpha\cos\theta\cos\theta'=\sin^2 2\alpha;
\]
\[
\text{but}\qquad\cos\theta=\mathrm{A}a\pm\mathrm{B}b+\mathrm{C}c,
\]
\[\qquad\qquad
\cos\theta'=\mathrm{A}a'+\mathrm{B}b'+\mathrm{C}c'.
\]

Putting these in the above, and multiplying
$\sin^2 2\alpha$ by $\mathrm{A}^2+\mathrm{B}^2+\mathrm{C}^2 (=1)$, we have
\[
(\mathrm{A}a+\mathrm{B}b+\mathrm{C}c)^2+(\mathrm{A}a'+\mathrm{B}b'+\mathrm{C}c')^2
\]
\[
-2\cos2\alpha(\mathrm{A}a+\mathrm{B}b+\mathrm{C}c)(\mathrm{A}a'+\mathrm{B}b'+\mathrm{C}c')
\]
\[
=\sin^2 2\alpha(\mathrm{A}^2+\mathrm{B}^2+\mathrm{C}^2),
\]
a homogeneous relation of the second order, and
therefore the proposition is true.

But besides establishing the truth of the proposition,
we are enabled immediately to find the
equation of the cone so traced, thus:

$x$, $y$, $z$, being written for A, B, C in the above
relation, making
\begin{align*}
 u & =ax+by+cz,\\
 u' & =a'x+b'y+c'z,\\
 r^2 & =x^2+y^2+z^2,
\end{align*}
we have for the equation in question
\[
r^2\sin^2 2\alpha=u^2+u'^2-2uu'\cos 2\alpha.
\]

The lines perpendicular to the planes $u=0$, $u'=0$,
are called focal lines for the following reason.
Consider $u'$ const.\ and $=p'$, then $p'$ is the perpendicular
from the origin on the plane $u'-p'=0$,
and $\sqrt{r^2-p'^2}$ is therefore the distance from a
point $x$,~$y$,~$z$, in the section of the cone made by
$u'-p'=0$ and the point in which $p'$ intersects
that plane.

The above equation gives
\[
(r^2-p'^2)\sin^2 2\alpha=(u-p'\cos 2\alpha)^2;
\]
\[
\therefore \quad \sqrt{r^2-p'^2}=  \text{ \textit{linear} function of } x,~y,~z.
\]

This is a property of the focus and the focus
only. Hence $p'$ passes through the foci of all sections
perpendicular to it. Similarly, a line perpendicular
to $u=0$ passes through foci of all
sections parallel to this plane.

Again, if the moveable line makes angles with
the fixed line, such that the product of their
cosines is constant, it will trace a cone of the
second order. The notation being as before, we have
\[
\cos\theta\cos\theta'=\mathrm{ const. }=n,
\]
\[
\text{or} \quad (\mathrm{A}a+\mathrm{B}b+\mathrm{C}c)(\mathrm{A}a'+\mathrm{B}b'+\mathrm{C}c')
=n(\mathrm{A}^2+\mathrm{B}^2+\mathrm{C}^2),
\]
a homogeneous equation of the second order.

This establishes the proposition and gives for
the equation of the cone $nr^2=uu'$, wherein $u=0$,
$u'=0$, are called cyclic planes for the following
reason.

Consider $u'$ constant and $=p'$, then
\[
r^2=\frac{1}{n} \cdot p'u,
\]
which being the equation to a sphere, the sections
parallel to $u'=0$, will be circular.

Similarly, sections parallel to $u=0$ are circular.

If two cones be supplementary to each other,
the cyclic planes of the one will be perpendicular
to the focal lines of the other.

Let the two supplementary cones be denoted
by
\[
\frac{x^2}{a^2}+\frac{y^2}{b^2}-\frac{z^2}{c^2}=0,
\]
\[
a^2x^2+b^2y^2-c^2z^2=0,
\]
in which $a$ is supposed greater than $b$.

Eliminate $x^2$ from the first by means of the
equation $r^2=x^2+y^2+z^2$, and it becomes
\[
r^2=a^2 \left\{ \left(\frac{1}{a^2}+\frac{1}{c^2} \right)z^2
- \left(\frac{1}{b^2}-\frac{1}{a^2} \right)y^2\ \right\}.
\]
Hence the two cyclic planes are
\[
\left(\frac{a^2+c^2}{c^2} \right)^{\frac{1}{2}}z
\pm \left(\frac{a^2-b^2}{b^2} \right)^{\frac{1}{2}}y=0,
\]
\begin{align*}
\mathrm{or}\quad u=\frac{b}{a}\left\{\frac{a^2+c^2}{b^2+c^2}\right\}^{\frac{1}{2}}z\;
+\; \frac{c}{a}\left\{\frac{a^2-b^2}{b^2+c^2}\right\}^{\frac{1}{2}}y=0,\\
u'=\frac{b}{a}\left\{\frac{a^2+c^2}{b^2+c^2}\right\}^{\frac{1}{2}}z\;
-\; \frac{c}{a}\left\{\frac{a^2-b^2}{b^2+c^2}\right\}^{\frac{1}{2}}y=0,\\
\end{align*}

Also eliminating $x^2$ from the other cone,
\[
r^2=\frac{1}{a^2}\{(a^2+c^2)z^2+(a^2-b^2)y^2\},
\]
but if $\cos2\alpha=\dfrac{b^2-c^2}{b^2+c^2}$, it is easily found that
this last equation may be put in the form
\[
r^2\sin^2 2\alpha=u^2+u'^2-2uu'\cos2\alpha.
\]

But we have shown that this is the form when
the perpendiculars to $u=0$, $u'=0$, are focal
lines. Hence the cyclic planes $u=0$, $u'=0$, of
the first cone are at right angles to the focal
lines of the second or supplementary cone.

Now, \textit{incidentally}, we have also proved that $2\alpha$
the sum or difference of angles which any side
of the second cone makes with its focal lines is
independent of $a$. Hence, if a system of cones
has the same vertex and axis, and be such that
sections made by a plane perpendicular to the
axis, all have the same major axis, the sum or
difference of the angles which any side of one
of the cones of the system makes with its focal
lines will be constant, not merely for the same
cone, but also for all the cones \smallskip of the system,
this sum or difference being $=2\tan^{-1}\dfrac{c}{b}$.
\smallskip

No one will fail to notice here the striking
analogy between spherical and plane conics.

Again, let the plane of $x$ $y$ be parallel to one
system of circular sections, and let the vertex of
the cone be the origin, and the line through the
centres of the circles be the axis of $z$, which line
will in general be inclined to plane $x$~$y$.

Take the axes of $x$,~$y$ perpendicular to each other.

The equation of the cone will be
\[
 x^2 + y^2 = n^2 z^2.
\]

Now taking $z$ constant and $nz=a$, suppose
we have $x^2 + y^2 = a^2$, and whatever property be
proved in plano respecting this circle, there will
necessarily be a corresponding one of the cone.
There will therefore be no greater analytical difficulty
in proving the conical property than that
in plano.

For example. ``If two tangent planes be drawn
to a cone of the second order such that their traces
on a cyclic plane are always inclined at the same
angle, the intersection of such planes will trace
out another cone of the second order having a
cyclic plane in common with the first cone''.

The analytical proof will be as follows.

The equations to the two tangent planes are
\begin{align*}
&x \cos(\theta + \alpha) + y \sin (\theta + \alpha) = nz,\\
&x \cos (\theta - \alpha) + y \sin (\theta - \alpha) =nz.
\end{align*}

Adding and dividing by $2\cos\alpha$,
\[
x \cos\theta + y\sin\theta = nz\sec\alpha.
\]

Subtracting $\qquad \qquad \quad x\sin\theta - y\cos\theta = 0$.

Taking sum of squares $x^2+y^2=n^2z^2\sec^2 \alpha$,
which is another cone having $x$~$y$ for a cyclic plane.

As another illustration take the following.

Let there be two given straight lines which intersect,
and let a plane perpendicular to the line
bisecting the angle between them be drawn, then
if two planes revolve about the given lines such
that their traces on the transversal plane include
a constant angle, the intersections of such planes
will trace out a cone of the second order which
shall have one of its cyclic planes parallel to
the transversal plane.

This proposition is in fact tantamount to proving
that if the base and vertical angle of a triangle
be constant, the locus of the vertex is a circle,
and it is from this plane proposition that Chasles
infers the conical one.

The following is the analysis.

The equations to the revolving planes have the
form
\begin{align*}
(x + mz)\cos(\theta + \alpha) + y\sin (\theta + \alpha) = 0, \\
(x - mz)\cos(\theta - \alpha) + y\sin (\theta - \alpha) = 0.
\end{align*}

The elimination of $\theta$ immediately gives
\[
(x^2 + y^2)\sin2\alpha = m^2z^2\sin2\alpha - 2mzy\cos2\alpha,
\]
which is a cone of the second order having circular
sections parallel to $x$~$y$.

It surely cannot be said that analysis loses any
of its usual advantages in the cases here adduced.
For my own part I always conclude that when
analysis does seem to lose any of its usual advantages,
the fault is not in the analysis, but in the
want of dexterity and clearness of analytical conception
in the analyst.

I am now about to make a remark to which
I think considerable importance is to be attached.

Whatever a plane problem may be, we may
always consider it as the result of one or more
relations between two variables or unknown quantities
$x$~and~$y$. \smallskip

Put $\dfrac{x}{z}$ for $x$ and $\dfrac{y}{z}$ for $y$, and we are sure
to have the corresponding conical problem.

\addcontentsline{toc}{section}{Mode of extending plane problems to conical problems}
Thus in the several investigations of Chapter II.
if we conceive \smallskip throughout $\dfrac{x}{z}$ and $\dfrac{y}{z}$ to be put for \smallskip
$x$ and $y$, and moreover consider $z$ constant \textit{until
the final result is obtained}, we shall have conical
properties corresponding to each of the plane properties.

\addcontentsline{toc}{section}{Enunciation of conical problems corresponding to
many of the plane problems in Chap. II.}
It will be sufficient to enunciate one or two,
as the reader will easily supply the rest.

If a system of cones touch the four planes of
a tetrahedral angle, the diameters of the several
individuals of that system conjugate to a given
fixed plane, will all lie in the same plane.

If a system of cones pass through the four
edges of a tetrahedral angle, the diameters of the
several individuals of that system conjugate to a
given fixed plane, will trace out another cone of
the second order.

If a system of cones pass through two of the
edges of a pentahedral angle and touch the two
opposite sides, the diameters of the several individuals
of that system conjugate to a given fixed
plane will trace out another cone of the second
order.

I have shown, therefore, how from any plane
problem a conical one may be deduced, and to
this class of conical problems I would propose
the name of plano-conical problems. There is
an equally extensive class arising from the intersection
of cones and concentric spheres, to which
the term sphero-conical problems might with propriety
be applied. These requiring a different
management the following illustrations are supplied.

``If through two fixed intersecting right lines
two rectangular planes be made to revolve, their
intersection will trace out a cone of the second
order passing through the fixed right lines and
having its cyclic planes at right angles to them.

This is another of Chasles' theorems.
\smallskip

Let $\dfrac{x}{\mathrm{A}}=\dfrac{y}{\mathrm{B}}=\dfrac{z}{\mathrm{C}}$
be the equations of the generating line.
\smallskip

Let the fixed lines be in the plane $x$~$z$ inclined
at an angle of $\alpha$ to axis of $z$.  By the property of
right-angled spherical triangles, we have $\cos2\alpha$
equal product of cosines of generating line with
fixed lines, or
\[
(\mathrm{C}\cos\alpha+\mathrm{A}\sin\alpha)
(\mathrm{C}\cos\alpha-\mathrm{A}\sin\alpha)=\cos2\alpha,
\]
where $\mathrm{A}^2+\mathrm{B}^2+\mathrm{C}^2=1$.

This equation is therefore
\[
\mathrm{A}^2\cos^2\alpha+\mathrm{B}^2\cos2\alpha-\mathrm{C}^2\sin^2\alpha=0,
\]
\[
\text{or} \qquad x^2\cos^2\alpha+y^2\cos2\alpha-z^2\sin^2\alpha=0.
\]

This is the equation to a cone, and on making
$y=0$, it reduces to
\[
x^2\cos^2\alpha-z^2\sin^2\alpha=0,
\]
\[
 \text{or} \quad x\cos\alpha-z\sin\alpha=0, \quad x\cos\alpha+z\sin\alpha=0,
\]
and, therefore, the surface passes through the two
fixed lines of which these are the equations.

Eliminating $y$ by the equation $r^2=x^2+y^2+z^2$,
\[
x^2\cos^2\alpha+(r^2-x^2-y^2)\cos2\alpha-z^2\sin^2\alpha=0,
\]
\[
 \text{or} \qquad r^2\cos2\alpha=z^2\cos^2\alpha-x^2\sin^2\alpha,
\]
and therefore the cyclic planes are
\[
z\cos\alpha+x\sin\alpha=0,
\]
\[
z\cos\alpha-x\sin\alpha=0,
\]
which are perpendicular to the given lines.

\addcontentsline{toc}{section}{Sphero-conical problems}
2. Let a system of cones of the second order
pass through the four edges of a tetrahedral angle,
to find the surface traced by the axes of each
individual of the system, or in other words, required
the locus of the spherical centres of a
system of spherical conics each passing through
four fixed points on the sphere.

Let $u$ and $v$ be two homogeneous functions of
$x$,~$y$,~$z$ of second order, so that $u=0$, $v=0$ may
represent two cones of that order. Suppose them
to intersect in four lines, then $u+\lambda v=0$ will
for different values of $\lambda$ represent all the cones
having the same vertex, and passing through the
same lines, for any point in any of the lines makes
$u=0$, $v=0$ separately, and therefore satisfies
the above equation.

Now in order to find the equations for the directions
of the axes we have, first considering $z$ as
a function of $x$ and $y$,
\[
\frac{du}{dx}+\frac{du}{dz}\cdot\frac{dz}{dx}
+\lambda\left\{\frac{dv}{dx}+\frac{dv}{dz}\cdot\frac{dz}{dx}\right\}=0,
\]
\[
\frac{du}{dy}+\frac{du}{dz}\cdot\frac{dz}{dy}+\lambda\left\{\frac{dv}{dy}+
\frac{dv}{dz}\cdot\frac{dz}{dy}\right\}=0.
\]

Putting $\quad \dfrac{dz}{dx}=-\dfrac{x}{z} \; , \;
\dfrac{dz}{dy}=-\dfrac{y}{z} \; , \quad$ \smallskip conditions which
insure the perpendicularity of the straight line
represented by the preceding with its conjugate
plane, and thus making it peculiar to the axes,
and then eliminating $\lambda$, the resulting equation is
\begin{align*}
  & x\left\{\frac{du}{dy}\cdot\frac{dv}{dz}-\frac{du}{dz}\cdot\frac{dv}{dy}\right\}\\[2mm]
+ & y\left\{\frac{du}{dz}\cdot\frac{dv}{dx}-\frac{du}{dx}\cdot\frac{dv}{dz}\right\}\\[2mm]
+ & z\left\{\frac{du}{dx}\cdot\frac{dv}{dy}-\frac{du}{dy}\cdot\frac{dv}{dx}\right\}=0.
\end{align*}

Now $u$, $v$ being homogeneous and of the second \smallskip
order, $\dfrac{du}{dx}$, \&c. \smallskip will be homogeneous and of the
first order. The preceding will therefore be homogeneous
and in general of the third order. Hence,
classifying the curves described on a spherical surface
by the orders of the equations of the concentric
cones by whose intersection with the
spherical surface they are produced, it will follow
that the locus of the centres of a system of spherical
conics of the second order passing through four
given points will be a spherical conic in general
of the third order.



\newpage
\addcontentsline{toc}{section}{Postscript, being remarks on a work by Dr.~Whewell, Master of
Trinity College, Cambridge, entitled, ``Of a Liberal Education
in general, and with particular Reference to the leading Studies
of the University of Cambridge''}

\begin{center}
\subsection*{POSTSCRIPT.}
\end{center}

It has seemed necessary to the Author, though he
is not a Member of the Senate of the Cambridge University,
to say a few words with reference to a work
which has lately appeared by Dr. Whewell, entitled
\lq\lq Of a liberal Education in general, and with particular
reference to the leading Studies of the University of
Cambridge.\rq\rq

What has been advanced in the preceding pages is
addressed chiefly to professed mathematicians, and is
intended to express the humble opinion of the Author
as to the supremacy of Analytical Mathematics. It is
by no means questioned or denied that appropriate ideas
both in Geometry, and other subjects, which are most
successfully carried on analytically, ought in the first
instance to be attained, and this by the study of such
works as Dr. Whewell classifies under the title of \lq\lq permanent
studies.\rq\rq\; If the ideas appropriate to each particular
branch be not first distinctly engraved on the
mind, it is to little purpose that we have recourse to
the aid of analysis; but having once possessed ourselves
of those ideas, analysis becomes a powerful instrument
for combining and generalizing to an extent which may
well be called infinite. The mode, however, of applying
it is a great matter.  In the hands of a person of little
skill it often leads to a labyrinth of perplexities and
false conclusions. Dr. Whewell, after asserting at p.~43
that \lq\lq Analytical reasoning is no sufficient discipline of
the reason, on account of the way in which it puts out
of sight the subject matter of the reasoning,\rq\rq\ further
adds, that \lq\lq The analyst does not retain in his mind, in
virtue of his peculiar processes, \textit{any apprehension} of the
differences of the things about which he is supposed to
be reasoning.\rq\rq

In answer to the first statement, it seems sufficient
to say that \lq\lq analytical reasoning\rq\rq\ cannot be charged
with putting out of sight the subject matter of the
reasoning on which it is employed, any more than the
pen, ink, and paper of the Author can be charged with
this concealment, for the former is just as much an instrument
as the latter. If B\oe vius strikes the lyre from
which Horace drew such sweet and lively strains, is the
lyre in fault because it does not reproduce them? And
in reference to the other remark, if an analyst \lq\lq does
not retain in his mind any apprehension of the differences
of the things about which he is supposed to be
reasoning,\rq\rq\  he is plainly a person who does not know
what he is about, and can only be \textit{supposed} to be reasoning
by those who know not how he ought to reason.

So strange does the whole of Dr. Whewell's reasoning,
in order to prove that \lq\lq Analysis is of little value
as a discipline of the reason for general purposes,\rq\rq\ appear
to us that we strongly suspect he has not had in
view that which mathematicians understand by the term
\lq\lq analysis,\rq\rq\  but that his object has been to \lq\lq counteract,
correct, and eradicate\rq\rq\  a vicious system of mathematical
education in the University, and which he ascribes
to the use of symbols. The Doctor is perfectly right as
to the existence of this vicious system, and very right,
as one of the guardians of education, to endeavour to
correct it, but most decidedly do we deny that \textit{analysis}
is to be blamed for it. The fact is, that for several
years past it has been the custom for incipient graduates,
after having passed the Senate House Examination with
more or less credit, to take pupils. Now those tutors
(many of them highly estimable men, and men of sterling
talent,) are often very inexperienced, probably most
of them thoroughly ignorant of mathematics only three
years previous to entering on their tutorial occupations.
But education, like every thing else, requires study,
thought, and experience. A young tutor may be a
highly talented individual, and yet a very bad teacher.
He has forgotten the difficulties experienced by himself,
and perhaps never known those experienced by others.
He cannot believe it possible that any being born with
reasoning faculties can stumble in going over the Pons
Asinorum, or fail to understand the Binomial Theorem,
and a thousand other minuti\ae. He therefore takes too
much for granted as to the state of knowledge of his
pupil, and all pupils are anxious to conceal rather than
display ignorance.

The tutor, therefore, taking it for granted that his
pupil is already possessed of the appropriate preliminary
ideas, ushers him too rapidly into the domains of analysis.
The pupil appears to progress rapidly, feels perfectly
satisfied both with himself and his tutor, and
soon begins to fancy that he \textit{may be} Senior Wrangler.
He can solve any quadratic, separate roots, draw tangents
and asymptotes, differentiate and integrate like
harlequin, and all this after having read only three
terms. But he cannot do the simplest deduction from
Euclid, has no idea of a geometrical limit, makes sad
bungling of a statical problem, and does not understand
Taylor's Theorem.

This state of things continuing, he is ushered into
Dynamics, Lunar and Planetary Theories, \&c. \&c. and
becomes ready for the Tripos, from which he emerges
last of the Senior Ops, retires from Cambridge, and is
puzzled all his life long to find out what is the use of
a University education.

Had such a youth been in the hands of an experienced
person and distinguished mathematician, who
would have taken care to \textit{educate} him properly, who
would have carefully \textit{completed the links} connecting geometrical
and algebraical reasoning, who would in every
analytical investigation have kept the ideas appropriate
to the subject of investigation constantly before the mind
of the pupil, who would have shown him in many cases
the identity of analytical and geometrical reasoning, and
that in all cases the former is as it were the sublimation
of the latter; then indeed it would not have been

\begin{center}
\lq\lq Parturiunt montes, nascitur ridiculus mus.\rq\rq
\end{center}

The preceding pages are intended to show to the
mathematicians of this country what may be done even
on a very common subject in the way of further developement
and generalization, by one, who while he is
employing analysis with all the skill of which he is
capable, never loses sight of appropriate ideas, but has
throughout the whole investigation those ideas vividly
in his mind. Had the work been principally intended
for learners, explanation would of course have been more
copious.

\vspace{20pt}
\textsc{R.M. Coll. Sandhurst,}\\
\indent 23\textit{rd Oct. }1846

\newpage

\small
\pagenumbering{gobble}
\begin{verbatim}

End of Project Gutenberg's Researches on curves of the 
second order, by George Whitehead Hearn

*** END OF THIS PROJECT GUTENBERG EBOOK RESEARCHES ON CURVES ***

*** This file should be named 17204-t.tex or 17204-t.zip ***
*** or                    17204-pdf.pdf or 17204-pdf.pdf ***
This and all associated files of various formats will be found in:
        http://www.gutenberg.org/1/7/2/0/17204/

Produced by Joshua Hutchinson, Jim Land and the Online
Distributed Proofreading Team at http://www.pgdp.net.
This file was produced from images from the Cornell
University Library: Historical Mathematics Monographs
collection.


Updated editions will replace the previous one--the old editions
will be renamed.

Creating the works from public domain print editions means that no
one owns a United States copyright in these works, so the Foundation
(and you!) can copy and distribute it in the United States without
permission and without paying copyright royalties.  Special rules,
set forth in the General Terms of Use part of this license, apply to
copying and distributing Project Gutenberg-tm electronic works to
protect the PROJECT GUTENBERG-tm concept and trademark.  Project
Gutenberg is a registered trademark, and may not be used if you
charge for the eBooks, unless you receive specific permission.  If
you do not charge anything for copies of this eBook, complying with
the rules is very easy.  You may use this eBook for nearly any
purpose such as creation of derivative works, reports, performances
and research.  They may be modified and printed and given away--you
may do practically ANYTHING with public domain eBooks.
Redistribution is subject to the trademark license, especially
commercial redistribution.



*** START: FULL LICENSE ***

THE FULL PROJECT GUTENBERG LICENSE PLEASE READ THIS BEFORE YOU
DISTRIBUTE OR USE THIS WORK

To protect the Project Gutenberg-tm mission of promoting the free
distribution of electronic works, by using or distributing this work
(or any other work associated in any way with the phrase "Project
Gutenberg"), you agree to comply with all the terms of the Full
Project Gutenberg-tm License (available with this file or online at
http://gutenberg.net/license).


Section 1.  General Terms of Use and Redistributing Project
Gutenberg-tm electronic works

1.A.  By reading or using any part of this Project Gutenberg-tm
electronic work, you indicate that you have read, understand, agree
to and accept all the terms of this license and intellectual
property (trademark/copyright) agreement.  If you do not agree to
abide by all the terms of this agreement, you must cease using and
return or destroy all copies of Project Gutenberg-tm electronic
works in your possession. If you paid a fee for obtaining a copy of
or access to a Project Gutenberg-tm electronic work and you do not
agree to be bound by the terms of this agreement, you may obtain a
refund from the person or entity to whom you paid the fee as set
forth in paragraph 1.E.8.

1.B.  "Project Gutenberg" is a registered trademark.  It may only be
used on or associated in any way with an electronic work by people
who agree to be bound by the terms of this agreement.  There are a
few things that you can do with most Project Gutenberg-tm electronic
works even without complying with the full terms of this agreement.
See paragraph 1.C below.  There are a lot of things you can do with
Project Gutenberg-tm electronic works if you follow the terms of
this agreement and help preserve free future access to Project
Gutenberg-tm electronic works.  See paragraph 1.E below.

1.C.  The Project Gutenberg Literary Archive Foundation ("the
Foundation" or PGLAF), owns a compilation copyright in the
collection of Project Gutenberg-tm electronic works.  Nearly all the
individual works in the collection are in the public domain in the
United States.  If an individual work is in the public domain in the
United States and you are located in the United States, we do not
claim a right to prevent you from copying, distributing, performing,
displaying or creating derivative works based on the work as long as
all references to Project Gutenberg are removed.  Of course, we hope
that you will support the Project Gutenberg-tm mission of promoting
free access to electronic works by freely sharing Project
Gutenberg-tm works in compliance with the terms of this agreement
for keeping the Project Gutenberg-tm name associated with the work.
You can easily comply with the terms of this agreement by keeping
this work in the same format with its attached full Project
Gutenberg-tm License when you share it without charge with others.

1.D.  The copyright laws of the place where you are located also
govern what you can do with this work.  Copyright laws in most
countries are in a constant state of change.  If you are outside the
United States, check the laws of your country in addition to the
terms of this agreement before downloading, copying, displaying,
performing, distributing or creating derivative works based on this
work or any other Project Gutenberg-tm work.  The Foundation makes
no representations concerning the copyright status of any work in
any country outside the United States.

1.E.  Unless you have removed all references to Project Gutenberg:

1.E.1.  The following sentence, with active links to, or other
immediate access to, the full Project Gutenberg-tm License must
appear prominently whenever any copy of a Project Gutenberg-tm work
(any work on which the phrase "Project Gutenberg" appears, or with
which the phrase "Project Gutenberg" is associated) is accessed,
displayed, performed, viewed, copied or distributed:

This eBook is for the use of anyone anywhere at no cost and with
almost no restrictions whatsoever.  You may copy it, give it away or
re-use it under the terms of the Project Gutenberg License included
with this eBook or online at www.gutenberg.net

1.E.2.  If an individual Project Gutenberg-tm electronic work is
derived from the public domain (does not contain a notice indicating
that it is posted with permission of the copyright holder), the work
can be copied and distributed to anyone in the United States without
paying any fees or charges.  If you are redistributing or providing
access to a work with the phrase "Project Gutenberg" associated with
or appearing on the work, you must comply either with the
requirements of paragraphs 1.E.1 through 1.E.7 or obtain permission
for the use of the work and the Project Gutenberg-tm trademark as
set forth in paragraphs 1.E.8 or 1.E.9.

1.E.3.  If an individual Project Gutenberg-tm electronic work is
posted with the permission of the copyright holder, your use and
distribution must comply with both paragraphs 1.E.1 through 1.E.7
and any additional terms imposed by the copyright holder.
Additional terms will be linked to the Project Gutenberg-tm License
for all works posted with the permission of the copyright holder
found at the beginning of this work.

1.E.4.  Do not unlink or detach or remove the full Project
Gutenberg-tm License terms from this work, or any files containing a
part of this work or any other work associated with Project
Gutenberg-tm.

1.E.5.  Do not copy, display, perform, distribute or redistribute
this electronic work, or any part of this electronic work, without
prominently displaying the sentence set forth in paragraph 1.E.1
with active links or immediate access to the full terms of the
Project Gutenberg-tm License.

1.E.6.  You may convert to and distribute this work in any binary,
compressed, marked up, nonproprietary or proprietary form, including
any word processing or hypertext form.  However, if you provide
access to or distribute copies of a Project Gutenberg-tm work in a
format other than "Plain Vanilla ASCII" or other format used in the
official version posted on the official Project Gutenberg-tm web
site (www.gutenberg.net), you must, at no additional cost, fee or
expense to the user, provide a copy, a means of exporting a copy, or
a means of obtaining a copy upon request, of the work in its
original "Plain Vanilla ASCII" or other form.  Any alternate format
must include the full Project Gutenberg-tm License as specified in
paragraph 1.E.1.

1.E.7.  Do not charge a fee for access to, viewing, displaying,
performing, copying or distributing any Project Gutenberg-tm works
unless you comply with paragraph 1.E.8 or 1.E.9.

1.E.8.  You may charge a reasonable fee for copies of or providing
access to or distributing Project Gutenberg-tm electronic works
provided that

- You pay a royalty fee of 20% of the gross profits you derive from
   the use of Project Gutenberg-tm works calculated using the method
   you already use to calculate your applicable taxes.  The fee is
   owed to the owner of the Project Gutenberg-tm trademark, but he
   has agreed to donate royalties under this paragraph to the
   Project Gutenberg Literary Archive Foundation.  Royalty payments
   must be paid within 60 days following each date on which you
   prepare (or are legally required to prepare) your periodic tax
   returns.  Royalty payments should be clearly marked as such and
   sent to the Project Gutenberg Literary Archive Foundation at the
   address specified in Section 4, "Information about donations to
   the Project Gutenberg Literary Archive Foundation."

- You provide a full refund of any money paid by a user who notifies
   you in writing (or by e-mail) within 30 days of receipt that s/he
   does not agree to the terms of the full Project Gutenberg-tm
   License.  You must require such a user to return or
   destroy all copies of the works possessed in a physical medium
   and discontinue all use of and all access to other copies of
   Project Gutenberg-tm works.

- You provide, in accordance with paragraph 1.F.3, a full refund of
   any money paid for a work or a replacement copy, if a defect in
   the electronic work is discovered and reported to you within 90
   days of receipt of the work.

- You comply with all other terms of this agreement for free
   distribution of Project Gutenberg-tm works.

1.E.9.  If you wish to charge a fee or distribute a Project
Gutenberg-tm electronic work or group of works on different terms
than are set forth in this agreement, you must obtain permission in
writing from both the Project Gutenberg Literary Archive Foundation
and Michael Hart, the owner of the Project Gutenberg-tm trademark.
Contact the Foundation as set forth in Section 3 below.

1.F.

1.F.1.  Project Gutenberg volunteers and employees expend
considerable effort to identify, do copyright research on,
transcribe and proofread public domain works in creating the Project
Gutenberg-tm collection.  Despite these efforts, Project
Gutenberg-tm electronic works, and the medium on which they may be
stored, may contain "Defects," such as, but not limited to,
incomplete, inaccurate or corrupt data, transcription errors, a
copyright or other intellectual property infringement, a defective
or damaged disk or other medium, a computer virus, or computer codes
that damage or cannot be read by your equipment.

1.F.2.  LIMITED WARRANTY, DISCLAIMER OF DAMAGES - Except for the
"Right of Replacement or Refund" described in paragraph 1.F.3, the
Project Gutenberg Literary Archive Foundation, the owner of the
Project Gutenberg-tm trademark, and any other party distributing a
Project Gutenberg-tm electronic work under this agreement, disclaim
all liability to you for damages, costs and expenses, including
legal fees.  YOU AGREE THAT YOU HAVE NO REMEDIES FOR NEGLIGENCE,
STRICT LIABILITY, BREACH OF WARRANTY OR BREACH OF CONTRACT EXCEPT
THOSE PROVIDED IN PARAGRAPH F3.  YOU AGREE THAT THE FOUNDATION, THE
TRADEMARK OWNER, AND ANY DISTRIBUTOR UNDER THIS AGREEMENT WILL NOT
BE LIABLE TO YOU FOR ACTUAL, DIRECT, INDIRECT, CONSEQUENTIAL,
PUNITIVE OR INCIDENTAL DAMAGES EVEN IF YOU GIVE NOTICE OF THE
POSSIBILITY OF SUCH DAMAGE.

1.F.3.  LIMITED RIGHT OF REPLACEMENT OR REFUND - If you discover a
defect in this electronic work within 90 days of receiving it, you
can receive a refund of the money (if any) you paid for it by
sending a written explanation to the person you received the work
from.  If you received the work on a physical medium, you must
return the medium with your written explanation.  The person or
entity that provided you with the defective work may elect to
provide a replacement copy in lieu of a refund.  If you received the
work electronically, the person or entity providing it to you may
choose to give you a second opportunity to receive the work
electronically in lieu of a refund.  If the second copy is also
defective, you may demand a refund in writing without further
opportunities to fix the problem.

1.F.4.  Except for the limited right of replacement or refund set
forth in paragraph 1.F.3, this work is provided to you 'AS-IS', WITH
NO OTHER WARRANTIES OF ANY KIND, EXPRESS OR IMPLIED, INCLUDING BUT
NOT LIMITED TO WARRANTIES OF MERCHANTIBILITY OR FITNESS FOR ANY
PURPOSE.

1.F.5.  Some states do not allow disclaimers of certain implied
warranties or the exclusion or limitation of certain types of
damages. If any disclaimer or limitation set forth in this agreement
violates the law of the state applicable to this agreement, the
agreement shall be interpreted to make the maximum disclaimer or
limitation permitted by the applicable state law.  The invalidity or
unenforceability of any provision of this agreement shall not void
the remaining provisions.

1.F.6.  INDEMNITY - You agree to indemnify and hold the Foundation,
the trademark owner, any agent or employee of the Foundation, anyone
providing copies of Project Gutenberg-tm electronic works in
accordance with this agreement, and any volunteers associated with
the production, promotion and distribution of Project Gutenberg-tm
electronic works, harmless from all liability, costs and expenses,
including legal fees, that arise directly or indirectly from any of
the following which you do or cause to occur: (a) distribution of
this or any Project Gutenberg-tm work, (b) alteration, modification,
or additions or deletions to any Project Gutenberg-tm work, and (c)
any Defect you cause.


Section  2.  Information about the Mission of Project Gutenberg-tm

Project Gutenberg-tm is synonymous with the free distribution of
electronic works in formats readable by the widest variety of
computers including obsolete, old, middle-aged and new computers.
It exists because of the efforts of hundreds of volunteers and
donations from people in all walks of life.

Volunteers and financial support to provide volunteers with the
assistance they need, is critical to reaching Project Gutenberg-tm's
goals and ensuring that the Project Gutenberg-tm collection will
remain freely available for generations to come.  In 2001, the
Project Gutenberg Literary Archive Foundation was created to provide
a secure and permanent future for Project Gutenberg-tm and future
generations. To learn more about the Project Gutenberg Literary
Archive Foundation and how your efforts and donations can help, see
Sections 3 and 4 and the Foundation web page at
http://www.pglaf.org.


Section 3.  Information about the Project Gutenberg Literary Archive
Foundation

The Project Gutenberg Literary Archive Foundation is a non profit
501(c)(3) educational corporation organized under the laws of the
state of Mississippi and granted tax exempt status by the Internal
Revenue Service.  The Foundation's EIN or federal tax identification
number is 64-6221541.  Its 501(c)(3) letter is posted at
http://pglaf.org/fundraising.  Contributions to the Project
Gutenberg Literary Archive Foundation are tax deductible to the full
extent permitted by U.S. federal laws and your state's laws.

The Foundation's principal office is located at 4557 Melan Dr. S.
Fairbanks, AK, 99712., but its volunteers and employees are
scattered throughout numerous locations.  Its business office is
located at 809 North 1500 West, Salt Lake City, UT 84116, (801)
596-1887, email business@pglaf.org.  Email contact links and up to
date contact information can be found at the Foundation's web site
and official page at http://pglaf.org

For additional contact information:
     Dr. Gregory B. Newby
     Chief Executive and Director
     gbnewby@pglaf.org

Section 4.  Information about Donations to the Project Gutenberg
Literary Archive Foundation

Project Gutenberg-tm depends upon and cannot survive without wide
spread public support and donations to carry out its mission of
increasing the number of public domain and licensed works that can
be freely distributed in machine readable form accessible by the
widest array of equipment including outdated equipment.  Many small
donations ($1 to $5,000) are particularly important to maintaining
tax exempt status with the IRS.

The Foundation is committed to complying with the laws regulating
charities and charitable donations in all 50 states of the United
States.  Compliance requirements are not uniform and it takes a
considerable effort, much paperwork and many fees to meet and keep
up with these requirements.  We do not solicit donations in
locations where we have not received written confirmation of
compliance.  To SEND DONATIONS or determine the status of compliance
for any particular state visit http://pglaf.org

While we cannot and do not solicit contributions from states where
we have not met the solicitation requirements, we know of no
prohibition against accepting unsolicited donations from donors in
such states who approach us with offers to donate.

International donations are gratefully accepted, but we cannot make
any statements concerning tax treatment of donations received from
outside the United States.  U.S. laws alone swamp our small staff.

Please check the Project Gutenberg Web pages for current donation
methods and addresses.  Donations are accepted in a number of other
ways including including checks, online payments and credit card
donations.  To donate, please visit: http://pglaf.org/donate


Section 5.  General Information About Project Gutenberg-tm
electronic works.

Professor Michael S. Hart is the originator of the Project
Gutenberg-tm concept of a library of electronic works that could be
freely shared with anyone.  For thirty years, he produced and
distributed Project Gutenberg-tm eBooks with only a loose network of
volunteer support.

Project Gutenberg-tm eBooks are often created from several printed
editions, all of which are confirmed as Public Domain in the U.S.
unless a copyright notice is included.  Thus, we do not necessarily
keep eBooks in compliance with any particular paper edition.

Most people start at our Web site which has the main PG search
facility:

     http://www.gutenberg.net

This Web site includes information about Project Gutenberg-tm,
including how to make donations to the Project Gutenberg Literary
Archive Foundation, how to help produce our new eBooks, and how to
subscribe to our email newsletter to hear about new eBooks.

*** END: FULL LICENSE ***

\end{verbatim}
\end{document}
---------------------------------------------------------
Below is appended the log from the most recent compile.
You may use it to compare against a log from a new
compile to help spot differences.
---------------------------------------------------------
This is e-TeX, Version 3.141592-2.2 (MiKTeX 2.4) (preloaded format=latex 2005.4.11)  4 DEC 2005 13:15
entering extended mode
**17204-t.tex
(17204-t.tex
LaTeX2e <2003/12/01>
Babel <v3.8a> and hyphenation patterns for english, french, german, ngerman, du
mylang, nohyphenation, loaded.
(C:\texmf\tex\latex\base\book.cls
Document Class: book 2004/02/16 v1.4f Standard LaTeX document class
(C:\texmf\tex\latex\base\bk11.clo
File: bk11.clo 2004/02/16 v1.4f Standard LaTeX file (size option)
)
\c@part=\count79
\c@chapter=\count80
\c@section=\count81
\c@subsection=\count82
\c@subsubsection=\count83
\c@paragraph=\count84
\c@subparagraph=\count85
\c@figure=\count86
\c@table=\count87
\abovecaptionskip=\skip41
\belowcaptionskip=\skip42
\bibindent=\dimen102
) (C:\texmf\tex\latex\amsmath\amsmath.sty
Package: amsmath 2000/07/18 v2.13 AMS math features
\@mathmargin=\skip43

For additional information on amsmath, use the `?' option.
(C:\texmf\tex\latex\amsmath\amstext.sty
Package: amstext 2000/06/29 v2.01
 (C:\texmf\tex\latex\amsmath\amsgen.sty
File: amsgen.sty 1999/11/30 v2.0
\@emptytoks=\toks14
\ex@=\dimen103
)) (C:\texmf\tex\latex\amsmath\amsbsy.sty
Package: amsbsy 1999/11/29 v1.2d
\pmbraise@=\dimen104
)
(C:\texmf\tex\latex\amsmath\amsopn.sty
Package: amsopn 1999/12/14 v2.01 operator names
)
\inf@bad=\count88
LaTeX Info: Redefining \frac on input line 211.
\uproot@=\count89
\leftroot@=\count90
LaTeX Info: Redefining \overline on input line 307.
\classnum@=\count91
\DOTSCASE@=\count92
LaTeX Info: Redefining \ldots on input line 379.
LaTeX Info: Redefining \dots on input line 382.
LaTeX Info: Redefining \cdots on input line 467.
\Mathstrutbox@=\box26
\strutbox@=\box27
\big@size=\dimen105
LaTeX Font Info:    Redeclaring font encoding OML on input line 567.
LaTeX Font Info:    Redeclaring font encoding OMS on input line 568.
\macc@depth=\count93
\c@MaxMatrixCols=\count94
\dotsspace@=\muskip10
\c@parentequation=\count95
\dspbrk@lvl=\count96
\tag@help=\toks15
\row@=\count97
\column@=\count98
\maxfields@=\count99
\andhelp@=\toks16
\eqnshift@=\dimen106
\alignsep@=\dimen107
\tagshift@=\dimen108
\tagwidth@=\dimen109
\totwidth@=\dimen110
\lineht@=\dimen111
\@envbody=\toks17
\multlinegap=\skip44
\multlinetaggap=\skip45
\mathdisplay@stack=\toks18
LaTeX Info: Redefining \[ on input line 2666.
LaTeX Info: Redefining \] on input line 2667.
)
(C:\texmf\tex\latex\amsfonts\amssymb.sty
Package: amssymb 2002/01/22 v2.2d

(C:\texmf\tex\latex\amsfonts\amsfonts.sty
Package: amsfonts 2001/10/25 v2.2f
\symAMSa=\mathgroup4
\symAMSb=\mathgroup5
LaTeX Font Info:    Overwriting math alphabet `\mathfrak' in version `bold'
(Font)                  U/euf/m/n --> U/euf/b/n on input line 132.
))
(C:\texmf\tex\latex\base\inputenc.sty
Package: inputenc 2004/02/05 v1.0d Input encoding file
 (C:\texmf\tex\latex\base\latin1.def
File: latin1.def 2004/02/05 v1.0d Input encoding file
))
(C:\texmf\tex\latex\graphics\graphicx.sty
Package: graphicx 1999/02/16 v1.0f Enhanced LaTeX Graphics (DPC,SPQR)

(C:\texmf\tex\latex\graphics\keyval.sty
Package: keyval 1999/03/16 v1.13 key=value parser (DPC)
\KV@toks@=\toks19
)
(C:\texmf\tex\latex\graphics\graphics.sty
Package: graphics 2001/07/07 v1.0n Standard LaTeX Graphics (DPC,SPQR)
 (C:\texmf\tex\latex\graphics\trig.sty
Package: trig 1999/03/16 v1.09 sin cos tan (DPC)
) (C:\texmf\tex\latex\00miktex\graphics.cfg
File: graphics.cfg 2003/03/12 v1.1 MiKTeX 'graphics' configuration
)
Package graphics Info: Driver file: dvips.def on input line 80.

(C:\texmf\tex\latex\graphics\dvips.def
File: dvips.def 1999/02/16 v3.0i Driver-dependant file (DPC,SPQR)
))
\Gin@req@height=\dimen112
\Gin@req@width=\dimen113
) (17204-t.aux)
LaTeX Font Info:    Checking defaults for OML/cmm/m/it on input line 74.
LaTeX Font Info:    ... okay on input line 74.
LaTeX Font Info:    Checking defaults for T1/cmr/m/n on input line 74.
LaTeX Font Info:    ... okay on input line 74.
LaTeX Font Info:    Checking defaults for OT1/cmr/m/n on input line 74.
LaTeX Font Info:    ... okay on input line 74.
LaTeX Font Info:    Checking defaults for OMS/cmsy/m/n on input line 74.
LaTeX Font Info:    ... okay on input line 74.
LaTeX Font Info:    Checking defaults for OMX/cmex/m/n on input line 74.
LaTeX Font Info:    ... okay on input line 74.
LaTeX Font Info:    Checking defaults for U/cmr/m/n on input line 74.
LaTeX Font Info:    ... okay on input line 74.
 [1

] [1


] (17204-t.toc
LaTeX Font Info:    Try loading font information for U+msa on input line 1.

(C:\texmf\tex\latex\amsfonts\umsa.fd
File: umsa.fd 2002/01/19 v2.2g AMS font definitions
)
LaTeX Font Info:    Try loading font information for U+msb on input line 1.
 (C:\texmf\tex\latex\amsfonts\umsb.fd
File: umsb.fd 2002/01/19 v2.2g AMS font definitions
)
[1

])
\tf@toc=\write3
 [2] [3]
Underfull \hbox (badness 10000) in paragraph at lines 196--203

 []

[1

] [2] [3] [4] [5]
File: images/fig1.eps Graphic file (type eps)
 <images/fig1.eps> [6

] [7]
Overfull \vbox (1.61685pt too high) has occurred while \output is active []


[8] [9] [10] [11]
Underfull \hbox (badness 10000) in paragraph at lines 710--713

 []

File: images/fig2.eps Graphic file (type eps)
<images/fig2.eps>
File: images/fig3.eps Graphic file (type eps)
 <images/fig3.eps> [12]
File: images/fig4.eps Graphic file (type eps)
 <images/fig4.eps> [13] [14] [15]
[16] [17

] [18] [19] [20] [21] [22] [23] [24]
File: images/fig5.eps Graphic file (type eps)
 <images/fig5.eps> [25] [26]
Overfull \hbox (7.44153pt too wide) detected at line 1444
[]\OMS/cmsy/m/n/10.95 f\OML/cmm/m/it/10.95 y \OMS/cmsy/m/n/10.95  \OML/cmm/m/i
t/10.95 m\OT1/cmr/m/n/10.95 (\OML/cmm/m/it/10.95 x \OT1/cmr/m/n/10.95 + 1)\OMS/
cmsy/m/n/10.95 gf\OML/cmm/m/it/10.95 y \OMS/cmsy/m/n/10.95  \OML/cmm/m/it/10.9
5 mx\OMS/cmsy/m/n/10.95 g[][] \OT1/cmr/m/n/10.95 + []\OMS/cmsy/m/n/10.95 f\OML/
cmm/m/it/10.95 y \OT1/cmr/m/n/10.95 + \OML/cmm/m/it/10.95 m\OT1/cmr/m/n/10.95 (
\OML/cmm/m/it/10.95 x \OT1/cmr/m/n/10.95 + 1)\OMS/cmsy/m/n/10.95 gf\OML/cmm/m/i
t/10.95 y \OT1/cmr/m/n/10.95 + \OML/cmm/m/it/10.95 mx\OMS/cmsy/m/n/10.95 g[][] 
\OT1/cmr/m/n/10.95 + 2\OML/cmm/m/it/10.95 m[]x\OT1/cmr/m/n/10.95 (2\OML/cmm/m/i
t/10.95 x \OT1/cmr/m/n/10.95 + 1)[][] = 0
 []

File: images/fig6.eps Graphic file (type eps)
<images/fig6.eps> [27] [28] [29] [30] [31] [32

] [33] [34]
Overfull \hbox (7.77496pt too wide) in paragraph at lines 1884--1887
[]\OT1/cmr/m/n/10.95 When $\OML/cmm/m/it/10.95 n \OT1/cmr/m/n/10.95 = 0$, the s
ur-face be-comes a cone, ori-gin at ver-tex, and $[] \OML/cmm/m/it/10.95  [] 
[] \OT1/cmr/m/n/10.95 =$
 []

[35] [36]
Overfull \hbox (4.85039pt too wide) in paragraph at lines 2009--2010
[]\OT1/cmr/m/it/10.95 Generation of Cones of the sec-ond de-gree, and their Sup
-ple-men-tary Cones. 
 []

[37] [38] [39] [40] [41] [42]
Overfull \hbox (0.43495pt too wide) in paragraph at lines 2370--2375
[]\OT1/cmr/m/n/10.95 Thus in the sev-eral in-ves-ti-ga-tions of Chap-ter II. if
 we con-ceive throughout
 []

[43] [44] [45] [46] [47] [48] [49] [1] [2] [3] [4] [5] [6] [7] [8] [9] [10]
(17204-t.aux)

 *File List*
    book.cls    2004/02/16 v1.4f Standard LaTeX document class
    bk11.clo    2004/02/16 v1.4f Standard LaTeX file (size option)
 amsmath.sty    2000/07/18 v2.13 AMS math features
 amstext.sty    2000/06/29 v2.01
  amsgen.sty    1999/11/30 v2.0
  amsbsy.sty    1999/11/29 v1.2d
  amsopn.sty    1999/12/14 v2.01 operator names
 amssymb.sty    2002/01/22 v2.2d
amsfonts.sty    2001/10/25 v2.2f
inputenc.sty    2004/02/05 v1.0d Input encoding file
  latin1.def    2004/02/05 v1.0d Input encoding file
graphicx.sty    1999/02/16 v1.0f Enhanced LaTeX Graphics (DPC,SPQR)
  keyval.sty    1999/03/16 v1.13 key=value parser (DPC)
graphics.sty    2001/07/07 v1.0n Standard LaTeX Graphics (DPC,SPQR)
    trig.sty    1999/03/16 v1.09 sin cos tan (DPC)
graphics.cfg    2003/03/12 v1.1 MiKTeX 'graphics' configuration
   dvips.def    1999/02/16 v3.0i Driver-dependant file (DPC,SPQR)
    umsa.fd    2002/01/19 v2.2g AMS font definitions
    umsb.fd    2002/01/19 v2.2g AMS font definitions
images/fig1.eps
images/fig2.eps
images/fig3.eps
images/fig4.eps
images/fig5.eps
images/fig6.eps
 ***********

 ) 
Here is how much of TeX's memory you used:
 1637 strings out of 95898
 16801 string characters out of 1195288
 81825 words of memory out of 1077410
 4570 multiletter control sequences out of 60000
 14782 words of font info for 55 fonts, out of 500000 for 1000
 14 hyphenation exceptions out of 607
 27i,12n,24p,321b,295s stack positions out of 1500i,500n,5000p,200000b,32768s

Output written on 17204-t.dvi (64 pages, 165724 bytes).





